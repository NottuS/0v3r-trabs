  In the last years, we have watched a big grow in the daily use of robotic systems. 
  More and more researchers in robotics have focus on the mobile robots developments, it makes that 
  robots be capable of move and interact with the environment, what generates a huge field of applications
  and new challenges like: localization, map building and the problem of simultaneous
  localization and map building(SLAM).

  This work proposes a localization system to mobile robots based on the article \cite{wifiRadar}, 
  which provides a localization method of a mobile terminal in a indoor environment. The method is 
  based on the received signal strength(RSS), which is very used in localization techniques 
  in the wireless sensors networks for the easy applicability. Other techniques to localization 
  in the wireless sensors networks are shown.
  
  The entire system was built on the Android plataform for the easy incorporation on robotics 
  due the big amount of sensors supported. 
  Some features of this operation system and the Android application are brief commented. 