\documentclass{beamer}

\usepackage[utf8]{inputenc}
\usepackage[brazil]{babel}

\usepackage{verbatim}

\usepackage{graphicx}

\usepackage{listings}
\usepackage{color}

\definecolor{mygreen}{rgb}{0,0.6,0}
\definecolor{mygray}{rgb}{0.5,0.5,0.5}
\definecolor{mymauve}{rgb}{0.58,0,0.82}
\definecolor{mybckg}{rgb}{0.95,0.95,0.95}
\lstset{
  backgroundcolor=\color{mybckg},  % choose the background color; you must add \usepackage{color} or \usepackage{xcolor}
  basicstyle=\footnotesize,       % the size of the fonts that are used for the code
  breakatwhitespace=false,         % sets if automatic breaks should only happen at whitespace
  breaklines=true,                 % sets automatic line breaking
  %captionpos=b,                    % sets the caption-position to bottom
  %commentstyle=\color{mygreen},    % comment style
  deletekeywords={...},            % if you want to delete keywords from the given language
  escapeinside={\%*}{*)},          % if you want to add LaTeX within your code
  extendedchars=true,              % lets you use non-ASCII characters; for 8-bits encodings only, does not work with UTF-8
  frame=single,                    % adds a frame around the code
  keepspaces=true,                 % keeps spaces in text, useful for keeping indentation of code (possibly needs columns=flexible)
  columns=flexible,
  %keywordstyle=\color{blue},       % keyword style
  %language=bash,                   % the language of the code
  morekeywords={*,...,Hello,LS,Update},            % if you want to add more keywords to the set
  %numbers=left,                    % where to put the line-numbers; possible values are (none, left, right)
  %numbersep=5pt,                   % how far the line-numbers are from the code
  %numberstyle=\tiny\color{mygray}, % the style that is used for the line-numbers
  rulecolor=\color{black},         % if not set, the frame-color may be changed on line-breaks within not-black text (e.g. comments (green here))
  %showspaces=false,                % show spaces everywhere adding particular underscores; it overrides 'showstringspaces'
  %showstringspaces=false,          % underline spaces within strings only
  %showtabs=false,                  % show tabs within strings adding particular underscores
  %stepnumber=2,                    % the step between two line-numbers. If it's 1, each line will be numbered
  %stringstyle=\color{mymauve},     % string literal style
  tabsize=2,                       % sets default tabsize to 2 spaces
  %title=\lstname                   % show the filename of files included with \lstinputlisting; also try caption instead of title
  literate=
  {á}{{\'a}}1 {é}{{\'e}}1 {í}{{\'i}}1 {ó}{{\'o}}1 {ú}{{\'u}}1
  {Á}{{\'A}}1 {É}{{\'E}}1 {Í}{{\'I}}1 {Ó}{{\'O}}1 {Ú}{{\'U}}1
  {à}{{\`a}}1 {è}{{\'e}}1 {ì}{{\`i}}1 {ò}{{\`o}}1 {ù}{{\`u}}1
  {À}{{\`A}}1 {È}{{\'E}}1 {Ì}{{\`I}}1 {Ò}{{\`O}}1 {Ù}{{\`U}}1
  {ä}{{\"a}}1 {ë}{{\"e}}1 {ï}{{\"i}}1 {ö}{{\"o}}1 {ü}{{\"u}}1
  {Ä}{{\"A}}1 {Ë}{{\"E}}1 {Ï}{{\"I}}1 {Ö}{{\"O}}1 {Ü}{{\"U}}1
  {â}{{\^a}}1 {ê}{{\^e}}1 {î}{{\^i}}1 {ô}{{\^o}}1 {û}{{\^u}}1
  {Â}{{\^A}}1 {Ê}{{\^E}}1 {Î}{{\^I}}1 {Ô}{{\^O}}1 {Û}{{\^U}}1
  {œ}{{\oe}}1 {Œ}{{\OE}}1 {æ}{{\ae}}1 {Æ}{{\AE}}1 {ß}{{\ss}}1
  {ç}{{\c c}}1 {Ç}{{\c C}}1 {ø}{{\o}}1 {å}{{\r a}}1 {Å}{{\r A}}1
  {€}{{\EUR}}1 {£}{{\pounds}}1,
  %basicstyle=\ttfamily
}

\useoutertheme{infolines} % add a footline 
\usetheme{Frankfurt}
%\setbeamertemplate{items}[square] % changes the markers, ball: 3-dimensional balls, circle: 2-dimensional (flat) circles, rectangle: rectangles, default: triangles 
%\setbeamertemplate{section in toc}[circle]
\setbeamertemplate{section in toc}[ball unnumbered]
\setbeamertemplate{enumerate item}[square]
\setbeamertemplate{itemize item}[triangle]
\setbeamertemplate{itemize subitem}[circle]
\setbeamertemplate{blocks}[rounded][shadow=true] % add rounded corners and a shadow to the box that surrounds the theorem
\setbeamertemplate{navigation symbols}{} % disable the drawing of navigation icons

\newlength{\wideitemsep}
\setlength{\wideitemsep}{\itemsep}
\addtolength{\wideitemsep}{10pt}
\let\olditem\item
\renewcommand{\item}{\setlength{\itemsep}{\wideitemsep}\olditem}

\title[Localização para Robôs Móveis]{Localização \textit{indoor} para robôs móveis}
\author[Edileuton H. de Oliveira]{Edileuton Henrique de Oliveira}
\institute[UFPR]{
  Departamento de Informática\\
  Universidade Federal do Paraná\\
  Bacharelado em Ciência da Computação\\
  Trabalho de Graduação\\
  Orientador: Prof. Eduardo Todt.
}

\usepackage{remreset}
\makeatletter
\@removefromreset{subsection}{section}
\makeatother


\begin{document}
%\maketitle

\begin{comment}
\frame{\titlepage}

\frame{
\frametitle{Sumário}
\tableofcontents
}
\end{comment}

\begin{frame}
  \titlepage
\end{frame}

\section*{Sumário}
\begin{frame}
\tableofcontents
\end{frame}

\setcounter{subsection}{1}

%\section{Introduçao}

\section{Robôs Móveis}

\begin{frame}{Robôs Móveis}
\begin{itemize}
 \item São sistemas incorporados no mundo real que se movem autonomamente e interagem com ele para realizar suas tarefas.
 \item Aplicações: limpeza, corte de grama, detectar riscos, explorações de ambientes desconhecidos, 
vigilância autônoma, resgatar sobreviventes e assistência a idosos ou pessoas com alguma incapacidade.
  \item Principais tarefas: planejamento de caminho, navegação,  
evitar obstáculos, controle de motores, \textbf{localização}, \textbf{construção e atualização de mapas}.

\end{itemize}
\end{frame}

\begin{frame}{Localização}

 \begin{itemize}
  \item A obtenção da posição e orientação do robô no mapa: \textbf{pose($x$, $y$, $\theta$)}.
  
  \item Identificação e subsequente triangulação por ângulos e por distância dos \textit{landmarks}.
  
  \item A identificação dos \textit{landmarks} 
 é feita fazendo observações do ambiente utilizando seus sensores(sonar, laser, câmera).
 
  \item O robô deve ser capaz de lidar com os erros de medição dos sensores, 
 incertezas e informações incompletas.
\end{itemize}
\end{frame}

\begin{frame}{Localização Probabilística}
\begin{itemize}
  \item Ao invés de calcular a posição exata, calcula-se a 
  probabilidade do robô estar numa certa posição.
  \item A localização probabilística nos da a distribuição de probabilidade de todas as possíveis 
 configurações do robô: \textit{\textbf{belief}}.
  \item Principais abordagens: \textbf{localização de Markov} e \textbf{filtro de Kalman}.
\end{itemize}
\end{frame}

\begin{frame}{Modelo de Movimento e Percepção}
\begin{itemize}
\item Quando o robô se movimenta a incerteza de sua posição aumenta. 
\item A cada deslocamento do robô faz a atualização da distribuição de probabilidade que pode ser dividida em 2 passos:
   \begin{itemize}
  \item \textbf{Atualização de ação}: o robô se move e estima sua posição através estimação da odometria: incerteza aumenta.
  \item \textbf{Atualização de percepção}: o robô faz uma observação usando seus sensores e corrige sua posição, 
  combinando seu \textit{belief} com a probabilidade das observações feitas: incerteza diminui.
 \end{itemize}
 
 \item Desse modo, a cada movimento o robô pode obter uma melhor estimativa de sua real posição.
 \end{itemize}
\end{frame}
 
 \begin{comment}
\begin{frame}{Localização de Markov}
\begin{itemize}
 \item Encontra caminhos mínimos de maneira rápida e sem ciclos % pois utiliza o Dijkstra
 \item Utiliza menos largura de banda % pois nao envia a tabela de roteamento inteira, envia somente o estado de seus enlaces 
 \item É possível utilizar diversas métricas no calculo do caminho mínimo % menor custo, maior vazão, maior confiabilidade
 \item Mantém mais de um caminho mínimo para um dado destino % caminhos devem possuir valores de métrica idênticos  % aumenta eficiencia pois possibilita a divisão do trafego 
\end{itemize}
\end{frame}

\begin{frame}{Localização filtro de Kalman}
\begin{itemize}
 \item Encontra caminhos mínimos de maneira rápida e sem ciclos % pois utiliza o Dijkstra
 \item Utiliza menos largura de banda % pois nao envia a tabela de roteamento inteira, envia somente o estado de seus enlaces 
 \item É possível utilizar diversas métricas no calculo do caminho mínimo % menor custo, maior vazão, maior confiabilidade
 \item Mantém mais de um caminho mínimo para um dado destino % caminhos devem possuir valores de métrica idênticos  % aumenta eficiencia pois possibilita a divisão do trafego 
\end{itemize}
\end{frame}
\end{comment}

\begin{frame}{Construção de Mapas}
A tarefa de mapeamento corresponde à atribuição de valores aos elementos do mapa, relacionando cada um a uma certa 
posição nele.
\begin{itemize}
 \item \textbf{Mapas Métricos}
    \begin{itemize}
    \item Mapa detalhado do ambiente, dividido em uma rede de células.
    \item Cada célula contem informações do ambiente como espaço ocupado por um objeto, 
    espaço livre para navegação, etc.
    \item Localização é mais precisa e menos sujeito a ambiguidades.
    \end{itemize}

 \item \textbf{Mapas Topológicos}
      \begin{itemize}
    \item  Marcos e relações são os elementos dos mapas topológicos, concentra-se apenas em pontos de interesse..
    \item Essas relações podem ser de vários tipos, tais como o deslocamento
relativo entre dois marcos, a existência de um caminho entre eles,
quantidade de energia gasto em um caminho entre marcos, etc.
    \end{itemize}
\end{itemize}
\end{frame}
\begin{comment}
\begin{frame}{Mapas Métricos}
Periodicamente, cada roteador:
\begin{itemize}
 \olditem verifica o estado de seus enlaces % através do envio de pequenas mensagens
 \olditem envia as informações do estado dos enlaces aos roteadores adjacentes % quando ocorre mudanças na rede, stop() quando a ocorre mais troca de mensagem
\end{itemize}
\end{frame}

\begin{frame}{Mapas Topológicos}
Periodicamente, cada roteador:
\begin{itemize}
 \olditem verifica o estado de seus enlaces % através do envio de pequenas mensagens
 \olditem envia as informações do estado dos enlaces aos roteadores adjacentes % quando ocorre mudanças na rede, stop() quando a ocorre mais troca de mensagem
\end{itemize}
\end{frame}
\end{comment}

\begin{frame}{SLAM}
\begin{itemize}
  \item O robô inicia a navegação em uma localização desconhecida e em um ambiente desconhecido.
  \item Constrói um mapa desse ambiente de maneira incremental.
  \item Utiliza esse mapa simultaneamente para calcular a sua localização.
  \item Principais abordagens:
  \begin{itemize}
  \item \textbf{Extended Kalman Filter SLAM}. 
  \item \textbf{Filtro de partículas}.
  \item \textbf{GraphSLAM}.
  \end{itemize}
\end{itemize}
\end{frame}

\section{Localização em Rede de Sensores sem Fio} % Descreva o NS-3

\begin{frame}{Rede de Sensores sem Fio}
\begin{itemize}
 \item Coleta informações em um campo monitorado.
 \item Nodos da rede podem ser estáticos ou móveis.
 \item O calculo da localização depende das informações localização dos nodos na rede.
 \item Geralmente requerem que os nodos conhecidos, estejam dentro 
	do raio de comunicação dos sensores em comum.
  \item A posição de um certo nodo pode ser obtida 
  a partir da posição ou direção dos nodos conhecidos por ele.
\end{itemize}
\end{frame}

\begin{frame}{TOA(\textit{Time of Arrival})}

\begin{itemize}
 \item Tempo medido em que um sinal chega a um
receptor pela primeira vez depois de emitido.
\item O valor medido é o tempo de transmissão somado ao
atraso do tempo de propagação.
\item Este atraso,$t_{i,j}$, entre a transmissão do sensor i e recepção do sensor j, é igual a distância entre o transmissor e o receptor, 
$d_{i,j}$, dividido pela velocidade de propagação do sinal, $v_{p}$.

\end{itemize}
\end{frame}

\begin{frame}{TDOA(\textit{Time Difference of Arrival of Two Different Signals})}

\end{frame}


\begin{frame}{AOA(\textit{Angle of Arrival})}
Os desafios de simular o OSPF já se iniciam na instalação
\begin{itemize}
 \item Três guias de instalação diferentes %  Uma para o somente o NS-3, uma para NS-3 + DCE, uma para NS-3 + DCE + Quagga
  \begin{enumerate}
   \olditem Somente o NS-3 % 
   \olditem O NS-3 e o DCE %
   \olditem O NS-3 e o DCE com suporte para o Quagga % 
  \end{enumerate}
  \begin{itemize}
   \olditem O DCE é um módulo do NS-3 que oferece suporte a aplicações reais % 
   \olditem O Quagga fornece implementações de protocolos de roteamento %
   \olditem O DCE e o Quagga são descritos com mais detalhes a seguir %
  \end{itemize}
 \item São necessários diversos pacotes % 
 \item Cada guia de instalação lista pacotes diferentes % 
 \item Alguns dos pacotes não são listados em nenhuma das guias % 
\end{itemize}
\end{frame}

\begin{frame}{RSS(\textit{Received Signal Strength})}
Os desafios de simular o OSPF já se iniciam na instalação
\begin{itemize}
 \item Três guias de instalação diferentes %  Uma para o somente o NS-3, uma para NS-3 + DCE, uma para NS-3 + DCE + Quagga
  \begin{enumerate}
   \olditem Somente o NS-3 % 
   \olditem O NS-3 e o DCE %
   \olditem O NS-3 e o DCE com suporte para o Quagga % 
  \end{enumerate}
  \begin{itemize}
   \olditem O DCE é um módulo do NS-3 que oferece suporte a aplicações reais % 
   \olditem O Quagga fornece implementações de protocolos de roteamento %
   \olditem O DCE e o Quagga são descritos com mais detalhes a seguir %
  \end{itemize}
 \item São necessários diversos pacotes % 
 \item Cada guia de instalação lista pacotes diferentes % 
 \item Alguns dos pacotes não são listados em nenhuma das guias % 
\end{itemize}
\end{frame}


\section{Android} % Detalhe os experimentos que vc fez no NS-3
\begin{frame}{Introdução}
Criar uma simulação no NS-3 é relativamente simples
\begin{itemize}
 \item Documentação detalhada %  
 \item Diversos exemplos % 
 \item Diversos mecanismos de monitoramento, porém alguns deles exijem determinado nível de conhecimento % 
 \item Oferece alguns desafios como, por exemplo, a documentação sobre falhas de enlace é escassa % 
\end{itemize}
\end{frame}

\begin{frame}{Arquitetura}
\begin{itemize}
 \item Em simulações sem falhas o roteamento ocorre de acordo com o esperado %  
 \item Quando ocorre uma falha os pacotes que utilizam o caminho falho não chegam aos seus destinos % 
 \item Não é calculado um novo caminho mínimo % portanto
 \item Os roteadores não são informados sobre a falha do enlace % porque
  \begin{enumerate}
   \olditem Não é implementado o protocolo \textit{Hello} para identificar a falha % 
   \olditem Não é implementado o protocolo \textit{Flooding} para informar todos os roteadores %
  \end{enumerate}
 \item Somente o protocolo \textit{Exchange} está implementado no NS-3 % não é exatamente o exchange mas tem a mesma função
\end{itemize}
\end{frame}

\begin{frame}{Componentes de uma Aplicação Android}
\begin{itemize}
 \item O NS-3 não fornece a implementação completa do OSPF % 
 \item O módulo \textit{Direct Code Execution} (DCE) passa a ser uma alternativa para simular o OSPF %  
 \item O DCE possui a implementação do protocolo OSPF feita pela \textit{Quagga Routing Suite} % 
 \item O DCE também permite utilizar executáveis do sistema operacional Linux % 
\end{itemize}
\end{frame}

\section{Implementação}
\begin{frame}{Conclusão}
\begin{itemize}
 \item Usar o DCE implica em trabalhar com a própria implementação do protocolo no sistema operacional % 
 \item Elimina a vantagem da simulação, na medida em que múltiplos detalhes devem ser considerados mesmo para testes simples %  
 \item É possível concluir que o trabalho futuro necessário, para permitir a continuidade deste trabalho, é completar a implementação do OSPF nativa do NS-3 % 
\end{itemize}
\end{frame}

\section{Conclusão}
\begin{frame}{Conclusão}
\begin{itemize}
 \item Usar o DCE implica em trabalhar com a própria implementação do protocolo no sistema operacional % 
 \item Elimina a vantagem da simulação, na medida em que múltiplos detalhes devem ser considerados mesmo para testes simples %  
 \item É possível concluir que o trabalho futuro necessário, para permitir a continuidade deste trabalho, é completar a implementação do OSPF nativa do NS-3 % 
\end{itemize}
\end{frame}

\end{document}
