\chapter{Conclusão}
\label{conclusao}
\begin{comment}
  Os robôs vem sendo muito utilizados na automatização de tarefas, e nesse trabalho podemos perceber que  uma tarefa simples, como deslocar um robô de um lugar a outro, 
  de forma autônoma, exige técnicas complexas de estatística e de várias áreas da computação como inteligência artificial, geometria computacional e processamento de 
  imagens.
  
    O conceito por trás desse trabalho é muito promissor, pois, a idéia de poder clicar em um mapa, ou falar o endereço onde você deseja ir,
    e seu celular ou tablet dirige seu carro para você, sem precisar tocar no volante, é extremamente interessante.
\end{comment}

\section{Trabalhos Futuros}
  As possibilidades de trabalhos futuros são enormes. Primeiro, há a possibilidade de adicionar um maior precisão a técnica 
  apresentada, por exemplo, levando em consideração as obstruções que interferem na força do sinal recebido,
  como o modelo de propagação de sinal proposto no artigo \cite{wifiRadar}. Nele são levados em consideração
  diversas variáveis para sua elaboração, sendo que a mais importante delas é a quantidade de obstáculos que estão 
  entre o transmissor de sinal(\textit{access point}) e o receptor (terminal móvel).
  Com base nesse parâmetro, uma equação da distância em função da força de sinal recebido
  poderá ser deduzida, que será útil para os cálculos da posição e rastreamento
  do terminal móvel. Ou ainda combinar técnicas baseadas em RSS com as tecnologias já largamente empregadas, tais como o GPS.
  
  Um outro problema importante do RSS \textit{fingerprinting} que não é tratado pelo sistema proposto é o da atualização dinâmica da 
  base de dados de \textit{fingerprints}. Em \cite{fingerPrint2} utiliza-se um modelo adaptativo para estimar o 
  RSS de cada \textit{access point} baseado na técnica de interpolação linear planar.
  
  E por ultimo, fazer com que o aparelho com Android possa controlar 
  um robô agindo como o cérebro do robô, onde este um
  se move e envia dados de seu sensores para o aparelho conforme for 
  requerido e assim tomando as decisões necessárias para
  que o robô haja de maneira autônoma. E com essa combinação, pode-se fazer algo como no SLAM, 
  onde os dados de RSS dos \textit{access points}, vão sendo coletados conforme o robô se locomove.
  E ainda pode-se aumentar precisão da técnica proposta nesse trabalho, utilizando 
  outros sensores do robô, pois um \textit{fingerprint} pode ser composto por outros dados
   além do RSS dos \textit{access points}, como laser, sonar, câmera, etc.
\begin{comment}
    As possibilidades de trabalhos futuros são enormes. Primeiro poderia implantar o modelo de propagação
de sinal, proposto no artigo \cite{wifiRadar}. Nele são levados em consideração
diversas variáveis para sua elaboração, sendo que a mais importante delas é a quantidade de obstáculos que estão 
entre o transmissor de sinal (Access Point) e o receptor (terminal móvel).
Com base nesse parâmetro, uma equação da distância em função da força de sinal recebido
poderá ser deduzida, que será útil para os cálculos da posição e rastreamento
do terminal móvel. Ou ainda combinar técnicas baseadas em RSS com as tecnologias já largamente empregadas, tais como o GPS.
    
    
    Poderia também, ao contrário do sistema proposto, tratar a possibilidade de não haver um mapa do ambiente, e utilizar técnicas que lidam com o SLAM, 
    como nos artigos \cite{construcaoMapas2}\cite{construcaoMapas}\cite{slam}. E utilizar um algoritmo de planejamento de trajetos mais robusto 
    como o proposto em \cite{voronoi}, que utiliza diagrama de Voronoi para criar um \textit{roadmap}. E ainda ao invés de usar um simples sonar para detectar
     obstáculos, utilizar a câmera do tablet.
     
     Podemos aumentar a escala e ao invés de utilizar um simples robô de 40 cm dentro de um prédio, utilizar o sistema de navegação em um carro, 
      em um ambiente maior \cite{googleCar}.
\end{comment}
