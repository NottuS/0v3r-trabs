\documentclass[12pt,a4paper]{ufpr}
% \usepackage[portuges,brazil]{babel}
% \usepackage[portuguese,brazil]{babel}

\usepackage[brazil]{babel}
\usepackage[utf8x]{inputenc}
\usepackage{amssymb,amsmath}
\usepackage{epsfig}
\usepackage{multirow}
\usepackage{lmodern}
\usepackage{textcomp} 
\usepackage[T1]{fontenc}
\usepackage{amssymb}
\usepackage{subfigure}
\usepackage{graphicx}
\usepackage{caption}
\usepackage{setspace}
\usepackage{ps-macros}
\usepackage{comment}
\usepackage{url}
% \usepackage{psfig}

\setcounter{secnumdepth}{3}    % n - numero de niveis de subsubsection numeradas
\setcounter{tocdepth}{3}       % coloca ate o nivel n no sumario

\title{Localiza\c{c}\~ao \textit{indoor} para rob\^os m\'oveis}
\author{Edileuton Henrique de Oliveira}
\advisortitle{Orientador} % ou Orientador
\advisorname{Eduardo Todt.}
\advisorplace{Departamento de Inform�tica, UFPR}  % departamento, instituicao
\city{Curitiba}
\year{2013}

\banca        % nao insira o nome do orientador, ja eh feito automaticamente
{Prof. Dr. Nome exmplo}{Instituto exemplo, UE}
{Prof. Dr. Nome exmplo}{Departamento de Inform�tica, UFPR}
{}{} % se nao houver deixe em branco {}{}
{}{}    % se houver um quarto membro na banca, inserir nome e instituicao

\defesa{21 de dezembro de 2013} % dia em que foi realizada a defesa da dissertacao


\begin{document}

%\makecapaproposta             % cria capa para proposta%
\makecapadissertacao           % cria capa para dissertacao de mestrado %
%\makerosto                     % cria folha de rosto para versao final da UFPR %
%\maketermo                     % cria folha com o termo de aprovacao da dissertacao%

%\singlespacing           % espacamento 1 - capa UFPR%
%\onehalfspacing          % espacamento 1/2 %
\doublespacing            % espacamento 2 - UFPR %

\pagestyle{headings}
\pagenumbering{roman}

%\chapter*{Agradecimentos}
%\input{agradecimentos.tex}          % possiu somente o texto

\tableofcontents

%\listoffigures         % se houver mais do que 3 figuras
%\addcontentsline{toc}{chapter}{\MakeUppercase{Lista de Figuras}}
%\newpage

%\listoftables        % se houver mais do que 3 tabelas
%\addcontentsline{toc}{chapter}{\MakeUppercase{Lista de Tabelas}}
%\newpage

\chapter*{Resumo}
\addcontentsline{toc}{chapter}{\MakeUppercase{Resumo}}
%Texto do resumo....
  A utilização de robôs no auxilio e automação de tarefas cresce a cada dia, e por isso 
  cresce também o numero de

  Este trabalho tem como objetivo implementar o método empírico sugerido no artigo \cite{wifiRadar},
  que prove a localização de um terminal móvel em um ambiente \textit{indoor}. 
  O método é baseado na força do 
  sinal recebido(\textit{Received Signal Strength} - RSS), que é muito explorada em
  em técnicas de localização em uma rede de sensores sem fio pela fácil aplicação. Outros objetivos 
  deste trabalho é apresentar conceitos relacionados a robótica, localização 
  em uma rede de sensores sem fio e a aplicação da plataforma Android em robótica e 
  algumas características deste sistema operacional.
  
\textbf{Palavras-chave:} RSS, Localização, Robótica, Android, WSNs.           % somente o texto
\newpage

%\chapter*{Abstract}
%\addcontentsline{toc}{chapter}{\MakeUppercase{Abstract}}
%\textit{With the significant increase of Internet users \cite{intStats}, it is also increasing concern the scalability to develop web applications. The MEAN Stack is a set of technologies focusing on the development of scalable web applications, which consists of the following tools: MongoDB, Express.js, AngularJS and Node.js.}

\textit{This paper aims to present the MEAN Stack as an option for the development of applications that depend on scalability, and show how is the integration of the components that comprise it.}

\textbf{Keywords:} MEAN, Javascript, Node.js, Angular.js, MongoDB, Express.js, Scalability.        % somente o texto
%\newpage


\pagenumbering{arabic}

\chapter{Introdu\c{c}\~ao}
\label{Introducao}


\section{Objetivo}
Este trabalho de graduação tem como foco apresentar um conjunto de ferramentas em Javascript que foram criadas para o desenvolvimento de aplicações web altamente escaláveis, demonstrando como estas ferramentas podem ser utilizadas em conjunto e quais suas vantagens e desvantagens em relação a outras tecnologias existentes. 
Um conjunto de ferramentas apresentadas é o MEAN Stack\footnote{Stack no sentido de pilha, no caso uma pilha de aplicações.}, que já vem sendo usados por profissionais que trabalham com aplicações web, e que basicamente é a utilização de 4 tecnologias baseadas em Javascript que são compostas pelo banco de dados NoSQL MongoDB, o framework back-end Express, o framework front-end AngularJS, e o servidor NodeJS. Além disto apresentamos outra ferramenta também baseada em Javascript que é a API para WebSockets chamada Socket.io.
Para efetuar alguns testes comparativos desenvolvemos uma aplicação usando MEAN Stack e o Socket.io  baseada em uma outra aplicação que é um jogo chamado Math Race\footnote{A idéia do jogo é realizar uma competição em tempo real para ver qual jogador acertar mais contas de matemáticas em determinado tempo.}.


\section{Organização do trabalho}
.
  
\chapter{Robôs Móveis}
\label{robosMoveis}

  Neste capitulo são discutidas as principais tarefas que o robô deve desempenhar
  para que ele possa concluir seu objetivos de navegação de forma autônoma e da melhor maneira possível, sendo elas: localização e construção de mapas.
  
\section{Localização}
A obtenção da posição e orientação (pose) do robô pode ser feita através da identificação e 
subsequente triangulação por ângulos e por distância dos \textit{landmarks} percebidos pelo robô e
 previamente conhecidos, onde a identificação dos \textit{landmarks} 
 é feita fazendo observações do ambiente utilizando seus sensores(sonar, laser, câmera).
 
 No cálculo da localização, o robô deve ser capaz de lidar com os erros de medição dos sensores, 
 incertezas (que tendem a crescer com o deslocamento do robô) 
 e informações incompletas \cite{localization1}.  Portanto, ao invés de calcular a posição exata, 
 o que pode ser feito é calcular a probabilidade do robô estar numa certa posição. Daí a necessidade da 
 localização probabilística, na qual, a incerteza é representada utilizando teoria da probabilidade: 
 ao invés de dar a melhor estimativa da configuração atual do robô, 
 a localização probabilística dá a distribuição de probabilidade de todas as possíveis 
 configurações do robô \cite{localization1}. Essa distribuição de probabilidade é chamada \textit{belief}.
 
 Quando o robô se movimenta a incerteza de sua posição aumenta. 
 Fazendo observações do ambiente e mesclando os dados obtidos com a estimação da odometria, 
 o robô pode combinar essas informações com o \textit{belief} anterior ao deslocamento. 
 Deste modo, a cada movimento o robô pode
 obter uma melhor estimativa de sua real posição, isso é chamado modelo de movimento e percepção.
 
 A cada deslocamento do robô é necessário fazer a atualização da distribuição de probabilidade de sua configuração. 
 A atualização pode ser dividida em 2 passos \cite{localization1}:
 \begin{itemize}
  \item Atualização de ação: o robô se move e estima sua posição através de seus sensores nesse passo a incerteza aumenta.
  \item Atualização de percepção: o robô faz uma observação usando seus sensores e corrige sua posição, 
  combinando seu \textit{belief} com a probabilidade de fazer essa observação.
  Aqui a incerteza diminui.
 \end{itemize}

 No cálculo da localização probabilística é necessário ter: a distribuição de probabilidade inicial, 
 o modelo de erro estatístico dos sensores e o mapa do ambiente.
 Há duas principais abordagens para solução da localização probabilística de robôs móveis: 
 localização de Markov e filtro de Kalman, descritos a seguir.  
 
 \subsection{Localização de Markov}
	A localização de Markov usa um \textit{grid} para representar a configuração do robô, onde cada célula do \textit{grid}
	contém a probabilidade de o robô estar nela \cite{localization1}. A distribuição de probabilidade associada à percepção 
	dos sensores também é discretizada. Durante as etapas de ação e percepção todas as células do \textit{grid} são atualizadas. 
	
	A atualização na etapa de ação é feita através da convolução da distribuição do \textit{belief} inicial com a distribuição
	da probabilidade da possível pose do robô após seu deslocamento, com a incerteza aumentada, segundo o modelo de deslocamento.
	Na etapa da percepção, a atualização e feita fazendo a convolução do \textit{belief} inicial com o modelo estatístico de erro
	dos sensores. A convolução pode ser feita através da regra de Bayes\cite{localization1}.
	
	A ideia principal da regra de Bayes é que a probabilidade de um evento A dado um evento B 
	depende não apenas do relacionamento entre os eventos A e B, 
	mas também da probabilidade marginal (ou "probabilidade simples") da ocorrência de cada evento:
	
	\begin{figure}[hb]
	\centering
	\includegraphics[scale=0.7]{images/bayes.png}
	\caption{Regra de Bayes.}
	\label{fig:topologia}
	\end{figure}
	
	Como todas as células são atualizadas nas etapas de ação e percepção, a localização de Markov requer grande 
	quantidade de processamento e memória.
 
 \subsection{Localização filtro de Kalman}
 
 
 Na localização filtro de Kalman, assume-se que a distribuição de probabilidade da configuração do robô e o modelo dos sensores,
 são continuas e gaussianas \cite{localization2}. Como a distribuição gaussiana é descrita através da média e variância, somente essas duas variáveis 
 são atualizadas nas etapas de ação e percepção. Portanto, seu custo computacional é pequeno se comparado com a localização de Markov.
 
O filtro de Kalman produz estimativas dos valores reais de grandezas medidas e valores associados predizendo um valor, 
estimando a incerteza do valor predito e calculando uma média ponderada entre o valor predito e o valor medido. 
O peso maior é dado ao valor de menor incerteza. As estimativas geradas pelo método tendem a estar mais próximas dos 
valores reais que as medidas originais pois a média ponderada apresenta uma melhor estimativa de incerteza que ambos os valores utilizados no seu cálculo.

 %inferências exatas sobre um sistema dinâmico linear, mas onde o espaço de estados das variáveis não observadas é contínuo e todas as variáveis, observadas e não observadas, apresentam distribuição normal (ou, frequentemente, distribuição normal multivariada).
 
  %A estimativa obtida desta forma é melhor que a estimativa obtida utilizando-se qualquer uma das medidas unicamente. Assim, é um algoritmo usual para fusão de sensores.
%Na execução dos cálculos para o filtro, a estimativa do estado e as covariâncias são representadas por matrizes,
%para tratar as múltiplas dimensões envolvidas num único passo do cálculo. 
%Desta forma, é possível representar as relações lineares entre diferentes variáveis de 
%estado (como posição, velocidade e aceleração) em qualquer um dos modelos de transição ou covariâncias, 
%onde o espaço de estados das variáveis apresentam distribuição normal.
  
O filtro de Kalman combina uma predição da posição atual do robô a com uma nova medida usando uma média ponderada. 
A ideia dos pesos é que valores com menor incerteza estimada sejam mais "confiáveis". 
Os pesos são calculados através da covariância, uma medida da incerteza estimada da predição do estado do sistema \cite{slam4}. 

O resultado da média ponderada é uma nova estimativa do estado, que se localiza entre o estado predito e o estado medido, 
apresentando uma melhor incerteza estimada que qualquer um dos dois unicamente. 
Este processo é repetido a cada fase, com a nova estimativa e sua covariância gerando a predição usada na próxima iteração. 
Isto significa que o filtro de Kalman funciona recursivamente e requer apenas a 
última estimativa - não o histórico completo - do estado de um sistema para calcular o próximo estado\cite{localization2}.
 
 \begin{comment}
  The probability distribution of both the
robot configuration and the sensor model is
assumed to be continuous and
Gaussian!
• Since a Gaussian distribution only
described through mean value
μ
and
variance
σ
2
, we need only to update
μ
and
Σ
2.
Therefore the computational cost is
very low!

 Localization is tracked from a known positions
and recovery from ambiguous situations and after
collision is not possible
\end{comment}


\section{Construção de Mapas}
A tarefa de mapeamento corresponde à atribuição de valores aos elementos do mapa, relacionando cada um a uma certa 
posição nele \cite{construcaoMapas2}. O tipo ideal de mapa
a ser utilizado ou construído por um robô móvel depende da tarefa e do ambiente onde
este esteja inserido. Também depende das características do robô, tais como os tipos
de sensores que ele tem, bem como a forma com ele se move\cite{construcaoMapas2}.  Há duas abordagens principais para a 
representação de mapas de ambiente \cite{construcaoMapas}.
São os mapas métricos e topológicos, que serão discutidos a seguir. 

\subsection{Mapas Métricos}
No caso da construção de mapas métricos, o objetivo é obter um mapa detalhado do ambiente, 
com informações sobre a forma e o tamanho dos objetos e os limites das
áreas livres para a navegação, tais como corredores, quartos, trilhas e estradas. Para
representar essa complexa e detalhada informação, o mapa é geralmente dividido em
uma densa rede de forma que cada célula contenha informações sobre a ocupação
desse espaço por um objeto e, possivelmente, outras características ambientais que
estão sendo mapeadas \cite{construcaoMapas2}. 

Com à alta densidade de informação nesses mapas, o resultado da localização é mais preciso e
menos sujeito a ambiguidades \cite{construcaoMapas2}. Como os mapas métricos 
demandam uma grande quantidade de informações, eles exigem muita capacidade de processamento e 
armazenamento.

Atualmente grande parte dos sistemas de mapeamento assume que o ambiente é estático
durante o mapeamento. Se uma pessoa anda dentro do alcance dos sensores do robô durante o
mapeamento, o mapa resultante conterá evidências a respeito de um objeto na localização
correspondente. Além disso, se o robô retornar para esta localização e varrer a área uma
segunda vez, sem a pessoa presente, a estimativa da posição será menos precisa, uma vez que as novas medições não
contêm nenhum vestígio correspondente àquela pessoa. A exatidão reduzida do mapa
resultante pode ter uma influência negativa no desempenho da localização robô.

\subsection{Mapas Topológicos}
No caso de mapas topológicos, o objetivo é construir uma estrutura relacional, normalmente um grafo, 
que de forma geral, registra informações sobre determinados
elementos ou locais do ambiente, chamados marcos \cite{construcaoMapas2}. 
Assim, marcos e relações são os elementos dos mapas topológicos.
Essas relações podem ser de vários tipos, tais como o deslocamento
relativo entre dois marcos, a existência de um caminho entre eles,
ou a quantidade de energia gasta em um caminho entre marcos. 
É fácil perceber que a informação contida no mapa topológico é
menos detalhada e distribui-se de forma dispersa, concentrando-se apenas em pontos de interesse.
Assim esta representação, é muito seletiva com relação à informação que ele registra \cite{construcaoMapas2}. 

O mapa topológico também demanda que o processamento
dos dados dos sensores do robô, sejam feitos de forma mais abstrata, 
a fim de extrair informações sobre as localidades mais relevantes que devem ser consideradas
como marcos.

\section{Problema de auto localização e mapeamento simultâneos}
  O problema de auto localização e construção de mapas de ambiente simultâneos (\textit{Simultaneous Localization and
Map Building - SLAM}) consiste em um robô autônomo iniciar a navegação
em uma localização desconhecida, em um ambiente desconhecido e então construir um mapa
desse ambiente de maneira incremental, enquanto utiliza o mapa simultaneamente para calcular a sua localização \cite{slam}.
 
 O SLAM tem sido tema de várias pesquisas na área de robótica. A grande vantagem do SLAM é que elimina a necessidade 
um conhecimento a priori do ambiente. Há três abordagens principais que são utilizadas no problema do SLAM. 
\begin{comment}
Pick natural scene features to serve as landmarks
(in most modern SLAM systems)

Range sensing (laser/sonar): line segments, 3D planes, corners

Vision: point features, lines, textured surfaces.

Key
: features must be distinctive & recognizable from different viewpoints

Keeping track of
changes in the environmen

Map can become inconsistent due to
erroneous measurements / motion drift

Their position uncertainty results
from the combination of the
measurement error with the
robot
pose
uncertainty


map becomes correlated with
the robot
pose
estimate

Robot moves again and its
uncertainty increases
(
motion
model)

Robot re
-
observes an old feature

Robot updates its position: the
resulting position estimate
becomes correlated with the
feature location estimates.

Robot‟s uncertainty shrinks and so
does the uncertainty in the rest of
the map
\end{comment}

A primeira e mais popular deles usa o filtro de Kalman estendido para resolver o SLAM (Extended Kalman Filter SLAM - EKF SLAM) \cite{slam2}. Ele
fornece uma solução recursiva para o problema da navegação e uma maneira de calcular
estimativas consistentes para a incerteza na localização do robô e nas posições dos marcos do
ambiente, com base em modelos estatísticos para o movimento dos robôs e observações
relativas dos marcos do ambiente\cite{slam}.

O EKF SLAM resume toda a toda experiencia obtida pelo robô em um vetor de estados estendido $Y$, compreendendo a pose do 
robô e posição dos marcos do mapa, e a matriz de covariância $P$\cite{slam2}. Quando o robô se desloca $Y$ e $P$ são atualizadas
usando o EKF. Os marcos do ambiente são extraídos do ambiente de sua nova posição. Em seguida, o robô tenta associar esses
marcos com os marcos previamente observados. A Reobservação dos marcos é usada para atualizar a posição do robô. 
Os marcos que não foram observados previamente, são adicionados no mapa de marcos.

\begin{comment}
When the odometry changes because the robot moves t
he uncertainty pertaining to the
robots new position is updated in the EKF using Odo
metry update. Landmarks are
then extracted from the environment from the robots
new position. The robot then
attempts to associate these landmarks to observatio
ns of landmarks it previously has
seen. Re-observed landmarks are then used to updat
e the robots position in the EKF.
Landmarks which have not previously been seen are a
dded to the EKF as new
observations so they can be re-observed later. All
these steps will be explained in the
next chapters in a very practical fashion relative
to how our ER1 robot was
implemented. It should be noted that at any point
in these steps the EKF will have an
estimate of the robots current position. 

Landmarks are features which can easily be re-obser
ved and distinguished from the
environment. These are used by the robot to find o
ut where it is (to localize itself).
One way to imagine how this works for the robot is
to picture yourself blindfolded. If
you move around blindfolded in a house you may reac
h out and touch objects or hug
walls so that you don’t get lost. Characteristic t
hings such as that felt by touching a
doorframe may help you in establishing an estimate
of where you are. Sonars and
laser scanners are a robots feeling of touch. 

Landmarks should be re-observable by allowing them
for example to be viewed
(detected) from different positions and thus from d
ifferent angles.
Landmarks should be unique enough so that they can
be easily identified from one
time-step to another without mixing them up. In ot
her words if you re-observe two
landmarks at a later point in time it should be eas
y to determine which of the
landmarks is which of the landmarks we have previou
sly seen. If two landmarks are
very close to each other this may be hard

The key points about suitable landmarks are as foll
ows:
Landmarks should be easily re-observable.
Individual landmarks should be distinguishable from
each other.
Landmarks should be plentiful in the environment.
Landmarks should be stationary. 

The first step is very easy. It is just an addition
of the controls of the robot to the old
state estimate. E.g. the robot is at point (x, y) w
ith rotation theta and the controls are
(dx, dy) and change in rotation is dtheta. The resu
lt of the first step is the new state of
the robot (x+dx, y+dy) with rotation theta+dtheta.
In the second step the re-observed landmarks are co
nsidered. Using the estimate of the
current position it is possible to estimate where t
he landmark should be. There is
usually some difference, this is called the innovat
ion. So the innovation is basically
the difference between the estimated robot position
and the actual robot position,
based on what the robot is able to see. In the seco
nd step the uncertainty of each
observed landmark is also updated to reflect recent
changes. An example could be if
the uncertainty of the current landmark position is
very little. Re-observing a
landmark from this position with low uncertainty wi
ll increase the landmark certainty,
i.e. the variance of the landmark with respect to t
he current position of the robot.
In the third step new landmarks are added to the st
ate, the robot map of the world.
This is done using information about the current po
sition and adding information
about the relation between the new landmark and the
old landmarks. 

http://ocw.mit.edu/courses/aeronautics-and-astronautics/16-412j-cognitive-robotics-spring-2005/projects/1aslam_blas_repo.pdf
http://www.asl.ethz.ch/education/master/mobile_robotics/year2012/Lecture10.pdf
\end{comment}

A segunda abordagem, chamada filtro de partículas SLAM, consiste em evitar a necessidade de estimativas absolutas da
posição e de medições precisas das incertezas para utilizar conhecimento mais qualitativo da
localização relativa dos marcos do ambiente e do robô para a construção do mapa global e
planejamento da trajetória\cite{construcaoMapas}. 

O filtro de partículas é um modelo matemático que representa a distribuição de probabilidade associada a um conjunto 
de partículas discretas. Uma partícula contém a estimativa da pose do robô com pesos associados(todos os pesos 
devem ser incrementados em 1) \cite{slam2}. Por período de amostragem, cada partícula é modificada de acordo com o
modelo do processo, incluindo a adição de ruído aleatório para simular o efeito do ruído
nas variáveis de estado e, então, o peso de cada partícula é reavaliado com base na última
informação sensorial. 

As partículas com pesos próximos de zero são descartadas e recriam-se novas partículas com base naquelas que sobram. Quando o
número efetivo de amostras está abaixo de um determinado limiar, geralmente calculado
com base em uma percentagem das M partículas, então a população das M partículas é
re-amostrada (resampling), eliminado-se probabilisticamente aquelas cujos pesos são pequenos
e duplicando aquelas com pesos elevados \cite{slam4}.

\begin{comment}
Particle filters:
mathematical models that represent probability distributions as a
set of discrete particles which occupy the state space.
Particle
= a point estimate of the state with an associated weight
(all weights should add up to 1)

Steps in Particle Filtering:

Predict
:
Apply motion prediction to each particle

Make measurements

Update
:
for each particle:
•
Compare the particle‟s predictions of measurements with the actual measurements
•
Assign weights such that particles with good predictions have higher weight

Normalize
weights of particles to add up to 1

Resample
: generate a new set of M particles which all have equal weights 1/M
reflecting the probability density of the last particle set.

FastSLAM
approach
[
Montemerlo
et al., 2002]
•
Solve state posterior using a
Rao
-
Blackwellized
Particle Filter
•
Each landmark estimate is represented by a 2x2 EKF.
•
Each particle is “independent” (due the factorization) from the others and
maintains the estimate of M landmark positions.
\end{comment}

O terceiro método é bastante amplo e afasta-se do rigor matemático do filtro de
Kalman, ou do formalismo estatístico, retendo uma abordagem essencialmente numérica ou
computacional para a navegação e para o problema de SLAM. Essa abordagem inclui o uso
de correspondência com marcos artificiais do ambiente\cite{construcaoMapas}, ela é chamada de GraphSLAM.
As posições do robô ao longo do tempo e os marcos correspondem a nós em um grafo \cite{slam2}. 
As informações odométricas entre posições consecutivas e os marcos vistos em diferentes posições equivalem as arestas do
grafo. O algoritmo é executado em 2 etapas. Na primeira etapa, o mesmo apenas acumula dados e
constrói o grafo. Na segunda etapa, o o grafo é rearranjado para acomodar
os dados obtidos. Diferente do EKF SLAM, o GraphSLAM estima a posição do robô durante todo o trajeto.

\begin{comment}
GraphSLAM
basic idea: SLAM can be interpreted as a sparse graph of nodes and constraints
between nodes

nodes:
robot locations and map
-
feature locations

edges:
constraints between
consecutive robot poses (given by the
odometry
input
u
) and
robot poses and the features
observed
from
these
poses
.

Key property: constraints are not to be thought as rigid constraints but as soft constraints
constraints acting like
springs

Solve full SLAM by relaxing these constraints
get the best estimate of the robot path and
the environment map by computing the
state of minimal energy
of this network

the
update
-
time
is
constant
and
the required
memory
is
linear
with the no. features
\end{comment}

\begin{comment}
\subsection{Problema de navegação e mapeamento completo do ambiente}
  No artigo \cite{cnn} - \textit{"A Bioinspired Neural Network for Real-Time Concurrent Map Building and Complete Coverage Robot Navigation in Unknown Environments"},
Luo e Yang propõem um modelo de mapeamento e navegação de robôs em ambientes desconhecidos e dinâmicos. Eles utilizam robôs limpadores para exemplificar o modelo. No 
qual o robô deve percorrer todo o local, para fazer sua limpeza.

  O mapa topológico do ambiente é discretizado em células, onde as células podem assumir o formato de quadrados, retângulos e triângulos. 
Utilizando quadrados e retângulos, o robô limpador têm 8 direções possíveis para deslocamento, enquanto que utilizando células triangulares, 
o robô possui 12 direções possíveis, assim permitindo que o robô navegue por caminhos mais curtos e flexíveis. 
Essas células formam uma rede neural, e cada célula pode assumir um estado: desconhecido, limpo, sujo e \textit{deadlock}. 

O mapeamento do ambiente é feito com o robô limpador percorrendo o ambiente em "ziguezague" e monitorando a atividade neural da rede.
A dinamicidade da atividade neural, representa as mudanças do ambiente. As areas sujas e com com obstáculos ficam no topo e no vale da atividade da rede neural, 
respectivamente. As áreas sujas atraem o robô, através da propagação da atividade neural. 
O planejamento do caminho para evitar colisões é feito em tempo real, baseado na dinamicidade da atividade neural da rede e na posição anterior do robô, assim ele 
é capaz de percorrer o ambiente, limpando as áreas sujas.
\end{comment}

\begin{comment}
\section{Planejamento de trajetos}
  O problema planejamento de trajetos, consiste em encontrar o melhor trajeto possível, 
  de um ponto de origem até um ponto de destino que não resulte em colisão com obstáculos \cite{voronoi}.
  Dependendo da quantidade de informação disponível sobre o ambiente, que pode ser completamente ou
   parcialmente conhecido ou desconhecido, as abordagens de planejamento podem variar consideravelmente.
   A definição de melhor trajeto também pode variar. Esse tópico vem ganhando mais relevância em áreas, além da robótica, como computação gráfica, sistemas de informações 
  geográficas e jogos \cite{planejamentoCaminhos}.
  
  A computação geométrica tem um papel especial dentro do desenvolvimento do planejamento de trajetos.
  As principais abordagens utilizando geometria computacional são: \textit{roadmap}, decomposição de células,
 e campo potencial. Elas serão apresentadas a seguir.

\subsection{\textit{Roadmap}}
      A técnica de \textit{roadmap} para planejamento de trajeto consiste em capturar a conectividade
      do espaço livre do ambiente em uma rede de curvas de uma dimensão, denominada \textit{roadmap} \cite{planejamentoTrajetorias}. 
      Uma vez que a rede é obtida, ela é vista como um conjunto de caminhos. O planejamento de trajeto pode ser feito 
      conectando as posições iniciais e finais do robô ao \textit{roadmap} e buscar neste um caminho entre estes 
      dois pontos.
      
      Vários métodos foram propostos, formando \textit{roadmaps} a partir de estruturas de computação 
      geométrica como diagrama de Voronoi e grafo de visibilidade.
      
\subsubsection{Diagrama de Voronoi}
Um diagrama de Voronoi é um estrutura geométrica que
representa informações de proximidade sobre uma série de
pontos ou objetos \cite{tecnicasNavegacao}. Dada uma série de sites ou objetos, o plano
bidimensional em que o robô se locomove geralmente contém
obstáculos, cada um destes obstáculos pode ser representado
por polígonos côncavos ou convexos. Para encontrar o
diagrama generalizado de Voronoi para esta coleção de
polígonos, podemos calcular o diagrama através de uma
aproximação, convertendo os obstáculos em uma série de
pontos \cite{voronoi}. 

  Primeiramente as faces dos polígonos são
subdivididas em uma série de pontos. O próximo passo é
calcular o diagrama de Voronoi para esta coleção de pontos.
Após o diagrama de Voronoi ser calculado, os segmentos do
diagrama que intersectam algum obstáculo são eliminados. E então 
utiliza-se uma algoritmo de busca em grafos para encontrar o melhor caminho.
Este método gera uma rota que na sua maior parte
permanece eqüidistante dos obstáculos, criando um
caminho seguro para o robô se locomover, apesar de não ser o
caminho mais curto. 
  O caminho encontrado pode ser melhorado como é mostrado em \cite{voronoi}, no qual pode-se 
  deixa-lo mais suave, retirando desvios desnecessários.
  
\subsubsection{Grafo de visibilidade}
  Um grafo de visibilidade é obtido gerando-se segmentos de
reta entre os pares de vértices dos obstáculos \cite{tecnicasNavegacao}. Todo o
segmento de reta que estiver inteiramente na região do espaço
livre é adicionada ao grafo. Para executar o
planejamento de trajetória, a posição atual e o objetivo são
tratados como vértices, isso gera um grafo de conectividade
onde utiliza-se algoritmos de procura para se encontrar um
caminho livre. O caminho mais curto que for encontrado no grafo de
visibilidade é o caminho ótimo para o problema especificado\cite{tecnicasNavegacao}.
Mas os caminhos encontrados no grafo podem tocar obstáculos.
Para utilizar um método de navegação com grafo de visibilidade é
necessário quer o mapa seja completo e bem definido.

\subsection{Decomposição de células}
  O método de decomposição de células consiste em dividir o espaço livre do robô em regiôes simples (células) \cite{planejamentoTrajetorias},
  de forma qur um caminho entre quaisquer duas configurações em uma mesma célula possa ser obtido. Um grafo
   não dirigido representando a relação de adjacência entre as células é construído. Os vértice que compôem este grafo são as células extraídas do espaço livre do robô. Há uma 
   aresta entre dois vértices se e somente se as células correspondentes a ele são adjacentes. Uma busca é feita nesse grafo 
   e seu resultado é uma sequência de células denominada canal. Um caminho continuo pode ser retirado do canal. A qualidade de caminho obtido, 
   é diretamente afetado pelo grau de decomposição de células.
  
\subsection{Campo potencial}
   A idéia por trás do método de campo potencial, é assinalar uma função similar ao potencial eletrostático
   para cada obstáculo, e então criar uma estrutura topológica de espaço livre do robô na forma de vales de potencial 
   mínimo \cite{voronoi}. O robô move-se em direção a objetivo, já que este um, gera um campo potencial que atrai o robô. Os obstáculos geram
    uma força de repulsão, que impede que o robô colida com obstáculos. Assim a direção do movimento do 
    robô é determinada pela força proveniente do campo potencial nesta determinada configuração. Um problema desse método, é que no caminho utilizado, o robô 
    pode ficar preso e nunca atingir o objetivo.
\end{comment}

\chapter{\textit{Wireless sensor networks} - WSNs}

Este capitulo aborda as principais técnicas de localização utilizando WSNs.

\label{wsn}
	WSNs é uma rede de sensores que coleta informações em um campo monitorado. Este tipo de rede tem sido utilizada em 
	várias aplicações, incluindo rastreamento de objetos, resgate, monitoramento de ambiente, entre outros \cite{omc}. Localização é um 
	tópico muito importante, pois, muitas aplicações das WSNs dependem do conhecimento das posições dos sensores. Na maioria das WSNs 
	os sensores são estáticos, mas a tendência em aplicações modernas é os sensores terem mobilidade.
	
	Há um conjunto de algoritmos de localização que utilizam WSNs. Onde o calculo da localização depende das informações 
	de localização dos nodos(que podem ser estáticos ou móveis) da rede. Esses algoritmos podem ser divididos em duas categorias: \textit{range-based}
	e \textit{range-free}. Algoritmos do tipo \textit{range-free} \cite{omc} \cite{wsnsLinear} geralmente requerem que os nodos conhecidos, estejam dentro 
	do raio de comunicação dos sensores em comum. Eles tem sido muito explorados pois não precisam de \textit{hardware} extra. Os algoritmos do tipo \textit{range-based},
	geralmente necessitam de algum tipo de \textit{hardware} especial, onde a posição de um certo nodo pode ser obtida a partir da posição ou direção dos nodos conhecidos por ele.
	
\section{TOA(\textit{Time of Arrival})}
	TOA é o tempo medido em que um sinal (rádio frequência, acústicos, ou outros) chega a um
receptor pela primeira vez depois de emitido\cite{gps}. O valor medido é o tempo de transmissão somado ao
atraso do tempo de propagação. Este atraso,$t_{i,j}$
, entre a transmissão do sensor i e recepção do sensor j, é igual a distância entre o transmissor e o receptor, 
$d_{i,j}$, dividido pela velocidade de propagação do sinal, $v_{p}$
. Esta velocidade para a rádio frequência é aproximadamente $10^{6}$  vezes mais rápida que a velocidade do som \cite{gps},
ou seja, 1 ms corresponde a 0,3 m na propagação sonora, enquanto para a rádio frequência, 1 ns corresponde a 0,3 m \cite{gps}.

O ponto chave das técnicas baseadas em tempo é capacidade do receptor de
estimar com precisão o tempo de chegada em sinais na linha da visão. Esta estimativa é prejudicada tanto pelo ruído 
aditivo quanto por sinais de multi percurso.

\section{TDOA(\textit{Time Difference of Arrival of Two Different Signals})}
  Os algoritmos de TDOA fazem a medição da diferença no tempo de recepção
de sinais de diferentes estações de base (Base Station's - BSs)), sem a necessidade de uma sincronização de
todos os participantes BSs e terminais móveis(MT)) \cite{tdoa}. Na verdade, a incerteza entre os tempos de referência
dos BSs e do MT pode ser removidas por meio de um cálculo diferencial. Por isso, apenas
os BSs envolvidos no processo de estimativa de localização deve ser bem sincronizados.

Para um par de BSs, digamos i e j, o TDOA,
$T_{ij}$, é dada por $T_{ij} TBSi - TBSj$, onde $TBSi$ e $TBSj$ são os tempos absolutos tomados 
pelo tempo de chegada nos BSs i e j, respectivamente. Supondo que o MT está em LOS (\textit{line-of-sight}) com ambos os BSs, i
e j, o MT deve situar-se em uma hipérbole. Uma segunda hipérbole, onde o
MT deve estar, pode ser obtido através de uma medição adicional de TDOA envolvendo
um terceiro BS. A posição do usuário pode ser identificada como o ponto de intersecção
das duas hipérboles. A solução do sistema de equações pode ser encontrado com um
método iterativo e minimização dos quadrados mínimos \cite{tdoa}.

O método dos quadrados mínimos procura encontrar 
o melhor ajuste para um conjunto de dados tentando minimizar a soma dos quadrados das diferenças 
entre o valor estimado e os dados observados (tais diferenças são chamadas resíduos).

Queremos estimar valores de determinada variável y. Para isso, consideramos os valores de outra variável x 
que acreditamos ter poder de explicação sobre y conforme a fórmula:

    y = $\alpha$ + $\beta$ x + $\varepsilon$

onde:
    \begin{itemize}
     \item $\alpha$: Parâmetro do modelo chamado de constante (porque não depende de x).
     \item  $\beta$: Parâmetro do modelo chamado de coeficiente da variável x.
     \item  $e$: Erro - representa a variação de y que não é explicada pelo modelo.
    \end{itemize}

Também temos uma base de dados com n valores observados de y e de x . 
O método dos mínimos quadrados ajuda a encontrar as estimativas de $\alpha$ e $\beta$.

O método dos mínimos quadrados minimiza a soma dos quadrado dos resíduos, ou seja, minimiza $\sum_{i=1}^n e_i^2$.

A ideia por trás dessa técnica é que, minimizando a soma do quadrado dos resíduos, encontraremos a e b que trarão 
a menor diferença entre a previsão de y e o y realmente observado.

\section{AOA(\textit{Angle of Arrival})}
A Localização baseada em AOA envolve a medição do ângulo de chegada de um sinal de uma BS
a um receptor ou vice-versa. Em qualquer um dos casos, uma única medida produz uma linha reta 
entre a BS e o receptor. A medida do ângulo de chegada com outra \textit{base station}
produzirá uma segunda linha recta e, a intersecção das duas linhas, vai fornecer a posição
do dispositivo\cite{aoa3}.
	\begin{figure}[hb]
	\centering
	\includegraphics[scale=0.5]{images/aoa.png}
	\caption{Anglel of Arrival\cite{aoa3}. }
	\label{fig:aoa}
	\end{figure}
	
      Existem duas maneiras nas quais sensores medem o AOA \cite{aoa}. O método mais
comum é usar um vetor de sensores. Neste caso, cada nó sensor é
composto de dois ou mais sensores individuais (microfones para sinais acústicos ou antenas de sinais de rádio frequência),
cujas posições com relação ao centro do nó são conhecidos. A AOA é estimada a partir
das diferenças de tempos de chegada de uma transmissão do sinal em cada um dos elementos do vetor de sensores.

  A segunda abordagem para a estimativa do AOA, usa a razão RSS entre duas (ou
mais) antenas direcionais localizadas no sensor. Duas antenas direcionais apontadas em
direções diferentes, de tal maneira que suas hastes se sobrepõem, podem ser usadas para
estimar a AOA partir da relação entre seus valores individuais RSS.

\begin{comment}
AOA is defined as the angle between the propagation
direction of an incident wave and some reference direction,
which is known as orientation.
Orientation
, defined as a fixed
direction against which the AOAs are measured, is represent
ed
in degrees in a clockwise direction from the North. When
the orientation is 0
◦
or pointing to the North, the AOA is
absolute, otherwise, relative. One common approach to obta
in
AOA measurements is to use an antenna array on each sensor
node.We assume that the beacons have
no information about their orientations and the unknowns ca
n
detect the AOA information between neighbor nodes by using
one of the above methods

http://citeseerx.ist.psu.edu/viewdoc/download?doi=10.1.1.134.5991&rep=rep1&type=pdf
\end{comment}



\section{RSS(\textit{Received Signal Strength})}	
	A localização baseada em RSS apresenta-se como uma das únicas a não necessitar de \textit{hardware} extra, e ser uma técnica
de fácil aplicação. 

  O funcionamento da RSS é baseado na medição do sinal no receptor e o valor
obtido indica a distância até o transmissor \cite{rss1}. Sem nenhuma interferência, a distância de dois nodos $i$ e $j$ pode ser 
relacionada com a força do sinal medido, através do modelo log-normal apresentado em \cite{rss2}:
\begin{comment}
In the wireless communication channel, the signal strength is related to the distance through the log-normal shadowing model [18]. 

Based on this model, the signal strength between the node Formula and the node Formula is given as follows; Based on this model, the signal strength between
the node i and the node j is given as follows
where
P0 is the path loss for the reference distance, and η is the path loss exponent, d(i,j) is the distance between node i and node j,
d0 is the reference distance, X σ is a Gaussian random variable of zero mean with standard deviation σ . If the obstructed interferencesdo not exist, the signal
strength rss(i,j) can be used directly for the localization procedure
	c
\end{comment}
	\begin{figure}[hb]
	\centering
	\includegraphics[scale=0.4]{images/CodeCogsEqn.png}
	\caption{Modelo Log-normal \cite{rss2}.}
	\label{fig:rssDist}
	\end{figure}
	
	Onde $P_0$ é a perda de percurso para a distância de referência,
$n$ á o expoente de perda de percurso, $d(i,j)$ é a distância entre os nodos $i$ e $j$,
$d_0$ é a distância de referencia, $X_\sigma$ é uma variável aleatória Gaussiana de média zero com desvio padrão $\sigma$.

Tal forma de medição por muitas vezes é
descartada, pois não leva em conta o ambiente onde está sendo aplicada a medição; logo,
corresponde a um resultado não confiável, obstruções e obstáculos proporcionam erros de grande relevância, 
sendo esse um dos maiores problemas dos métodos de localização baseados em RSS, várias técnicas foram propostas para 
contornar esse problema como pode ser visto em \cite{wifiRadar}\cite{rss1}\cite{wsnsLinear}\cite{multAgent} \cite{rss2}.

A figura abaixo mostra como a estimativa da real posição de um nodo é afetado pela interferências de obstrução:
	\begin{figure}[ht]
	\centering
	\includegraphics[scale=0.5]{images/rsserror.png}
	\caption{Influência de um obstáculos na localização de um nodo\cite{rss1}. }
	\label{fig:rsserror}
	\end{figure}

   Em \cite{wifiRadar} - \textit{"Radar: an in-building RF-based user location and tracking system"}, Bahl e Padmanabhan propõem o método empírico,
   para localização baseada em RSS \textit{indoor}. O método consiste em realizar uma série de medições da força de sinal das estações bases, 
   e a partir dessas medições, a localização pode ser determinada fazendo uma triangulação dos dados coletados. 

  No artigo \cite{wifiRadar}, para obter a posição de um nodo, são medidas as forças do sinal wifi de 3 \textit{Access Points} (rss1, rss2, rss3), 
utiliza-se a distância Euclidiana, para fazer a comparação com os dados previamente coletados $sqrt((rss1-rss1')^{2}+(rss2-rss2')^{2}+(rss3-rss3')^{2})$, 
e assim encontrar o ponto que mais se aproxima dos parâmetros coletados naquele local. 

\input{android.tex}
\chapter{Implementação}	 
\label{implementacao}
  
\section{Detalhes da implementação}

  O aplicativo implementado nesse trabalho é composto essencialmente por três  classes: MainActivity, 
  Map e WifiInterface. 
  
  A comunicação entre as três classes é feito através da classe Handler(citar) que permite o envio de 
  mensagens entre \textit{threads} e componentes. As principais operações disponíveis no aplicativo não 
  são executadas na \textit{thread} principal, pois, essas operações podem levar muito tempo para serem 
  concluída, assim evitando o problema já discutido anteriormente no capitulo sobre Android,
  de travar a execução da \textit{thread} principal.
  
  \subsection{MainActivity}
  A classe MainActivity é responsável por receber as entradas do usuário e exibir os resultados calculados.
  Ela também é responsável por instanciar as duas outras classes. Há três funções disponíveis: 
  \begin{itemize}
   \item Scan: Obtém o \textit{fingerprint} da força do sinal wifi dos APs na coordenada informada.
   \item Save Table: Salva a tabela de \textit{fingerprints} em um arquivo.
   \item Get Position: Obtém a provável posição do aparelho. 
  \end{itemize}

  A MainActivity cria novas \textit{threads} para executar cada uma das operações requerida pelo o usuário.
  
  \subsection{WifiInterface}
  
  Esta classe centraliza todas as operações envolvendo wifi como verificar e habilitar a conectividade
  com o wifi do aparelho e obter as informações dos APs(força do sinal, nome do AP, endereço, etc) 
  que estão no alcance do aparelho. Para obter essas informações, o objeto dessa classe
  instancia um BroadcastReceiver(citar) que escuta o modulo wifi do aparelho e repassa 
  as informações obtidas através do Handler para o componente que requisitou elas.
  
  \subsection{Map}
  Esta classe é o núcleo da aplicação, ela carrega e armazena o mapa de \textit{fingerprints} de RSS que estão 
  guardadas em arquivo. 
  O \textit{fingerprints} é composto 
  pela coordenada informada pelo usuário e as informações obtidas pela classe WifiInterface. 
  Com o mapa de \textit{fingerprints} essa classe executa o algoritmo descrito a seguir para disponibilizar 
  a atual posição do aparelho.
   
  \subsection{Algoritmo de Localização baseado em \textit{fingerprinting}}
  
  O algoritmo de localização aqui implementado é dividido em duas etapas: aprendizagem e localização. Na etapa de aprendizagem, 
  onde o aplicativo guarda o \textit{fingerprint} dos APs da coordenada informada,
  e salva em uma tabela. O \textit{fingerprint} é composto da média de 8 amostras do RSS de cada AP, média é utilizada 
  devido a grande oscilação do RSS de um AP:
  
  As amostras foram obtidos do primeiro andar do departamento de informática da Universidade Federal do Paraná devido a 
  grande quantidade de APs disponíveis(até treze APs em uma coordenada), a imagem abaixo mostra os locais das amostras retiradas.
  
  A etapa de onde é realizada a localização em si, é dividido em três fases. Primeiramente obtêm-se o \textit{fingerprint}
  o aparelho com Android, da mesma forma feita na etapa de aprendizagem. 
  
  Na segunda fase, para cada três APs do \textit{fingerprint}, utiliza-se a distância Euclidiana, para fazer a comparação com os 
  dados previamente coletados $sqrt((rss1-rss1')^{2}+(rss2-rss2')^{2}+(rss3-rss3')^{2})$, 
e assim encontrar o ponto que mais se aproxima dos parâmetros coletados naquele local. 
Assim obtêm-se um conjunto de pontos que são utilizada na próxima fase.

 
  \begin{comment}
  -Criação da tabela:
    -para a coordenada em questão são obtidas 8 amostras da força do sinal wifi, e é feita a 
    média de cada força de sinal wifi de cada AP.
	- Pois O RSS varia muito(exibir exemplo).
  -Calculo da posição(3 fases):
    - Permite obter a posição do aparelho mesmo sem um fingerprint da posição do aparelho.
	- com as triangulações obtem-se o pontos que se aproxima da posição real.
    -obtem se 8 amostras do APs wifis e faz-se média.
    - para cada 3 APs obtem-se um ponto do mapa.
	- Esse ponto é obtido calculando a menor distância euclidiana entre os APs na amostragem e 
	e os dados salvos na tabela.
    - Com o esse conjunto de pontos, calcula-se o mmq com a biblioteca do apache(citar).
      -Onde obtem se o melhor ponto que resume os pontos obtidos.
    -Colocar imagem.
    \end{comment}
\section{Análise e Resultados Obtidos}

\begin{comment}
  Os experimentos serão realizados no departamento de informática da Universidade Federal do Paraná, 
  devido a grande quantidade de Access Points disponíveis e a disponibilidade da planta do departamento. 
  A implementação do sistema de navegação será dividida em duas etapas:
  \subsection{Etapa 1}
  
  Uma aplicação escrita em C será utilizada para criação do mapa topológico do departamento. Essa aplicação recebe como entrada:
  \begin{itemize}
    \item Uma imagem contendo o mapa.
    \item Tamanho da célula.
  \end{itemize}
 
  A imagem será dividida em células. Em todas as células, serão aplicadas a função de transformação de Hough \cite{openCV}, da biblioteca OpenCV para detecção de linhas. 
  Esta aplicação devolve uma matriz, onde célula possui uma valor que indica o grau de incerteza de haver um obstáculo: 0 (sem obstáculo) ou 4(com obstaculo). 
  Essa matriz será embarcada na aplicação Android da etapa 2.
  
  O tamanho ideal da célula, será definido via experimentos. Em \cite{cnn} o tamanho da célula tem o tamanho do robô. O artigo \cite{dlite} sugere que o tamanho da célula deve 
  comportar todo o obstáculo, lembrando que a quantidade de células pode influênciar no tempo de processamento do planejamento da trajetória do robô \cite{voronoi}.
  
    Uma segunda aplicação Android, será utilizada para coletar amostras, como descrito no método empírico do artigo\cite{wifiRadar}. As tabelas resultantes dessa aplicaçao, também serão 
  embarcada na aplicação Android da etapa 2. As linha da tabela conterão as coordenadas, RSS e nome do AP da amostra daquele local. Vale ressaltar, que número de amostras tem grande impácto 
  na precisão do calculo da localização \cite{wifiRadar}.
  
  \subsection{Etapa 2}
	A aplicação Android dessa etapa, possui 2 componentes principais: uma \textit{Acitivity} \cite{activity} que fará a interação com o usuário, 
	e um Service \cite{service} que fará o processo de navegação do robô. O robô e o tablet, se comunicarão via \textit{bluetooth},
	ou seja, para o funcionamento do sistema, o tablet deve estar pareado com o robô.

	A \textit{Acitivity} irá obter a posição do clique em relação à imagem do mapa, e obter as coordenadas da posição de destino
	 do robô e repassar para o Service. O usuário deve clicar no botão conectar para que a aplicação possa iniciar uma conexão \textit{bluetooth} com o robô.
	Haverá uma imagem com o mapa do ambiente, o usuário deve apontar o local no qual o robô deve atingir. E então clicar no botão iniciar.
	
	Um Service será iniciado, e executará os seguintes passos:
	\begin{itemize}
	  \item O mapa topológico será carregado na memória.
	  \item A posição do robô será calculada, medindo o RSS\cite{wifiRss} dos APs e aplicando o método descrito em \cite{wifiRadar}.
	  \item Em seguida será aplicado o algoritmo A*\cite{aestrela} na matriz gerada na etapa 1. A função h (heurística) levará em consideração o valor da 
	célula(celulas com valor inferior a 3 serão descartados do caminho) e a sua distância euclidiana, em relação a célula destino. 
	  \item Com o trajeto definido, um comando(vetor de bytes indicando distância e direção) será enviado ao robô.
	  \item O Service aguarda o recebimento dos dados do robô.
	  \item Se nenhum obstáculo é encontrado um novo comando é enviado, e processo se repete, até que o robô atinja seu objetivo.
	  \item Se o robô encontrar um novo obstáculo, a célula onde este foi encontrado é incrementado em 1. 
	  Então o processo de descoberta de trajeto é iniciado novamente.
	  \item Se por acaso o robô verificar está célula de novo e não houver obstáculo, o valor da célula é decrementado em 1.
	  \item Se o robô ficar preso, então é adotado um procedimento descrito em \cite{dlite},
	  muros virtuais são criado, para que o robô evite este caminho em uma segunda passagem.
	\end{itemize}
	
	 Aplicação Arduino faz a interface com os motores, o sonar e o adaptator \textit{bluetooth}.
	 Basicamente, ela executa o comando recebido, e então envia os dados coletados pelo sonar e aguarda um novo comando.
\end{comment}
\chapter{Conclus\~ao}
\label{Conclusao}

%O protocolo \textit{OSPF} � composto por tr�s subprotocolos. Os tr�s s�o essenciais para o funcionamento correto do \textit{OSPF}. Entretanto o simulador \textit{NS-3} possui a implementa��o de somente um desses protocolos, o protocolo \textit{Exchange}. Este � o protocolo respons�vel pela sincroniza��o de dados entre os roteadores. Ele � essencial para que cada roteador mantenha uma representa��o local da topologia da rede id�ntica aos demais. Deste modo todos os roteadores encontram caminhos m�nimos iguais. Com isso � poss�vel fazer o roteamento dos pacotes sem que ocorram ciclos. Por�m sem a implementa��o dos outros portocolos que comp�em o \textit{OSPF} a rede n�o suporta falhas. Caso um enlace falhe os roteadores da rede n�o ser�o informados. Portanto n�o � calculado um novo caminho m�nimo e os caminhos que utilizam o enlace falho se tornam inutiliz�veis. Para realizar simula��es de redes din�micas � necess�rio utilizar o m�dulo \textit{DCE}. Este m�dulo fornece o protocolo \textit{OSPF} implementado pela \textit{Quagga Routing Suite}. Deste modo nas simula��es com falhas os roteadores s�o informados e calculam novos caminhos m�nimos.

%A topologia da rede utilizada nas simula��es � pequena e simples. Para montar um base de dados sobre o comportamento do protocolo \textit{OSPF} � importante realizar simula��es em redes maiores e complexas. Al�m disso, � interessante estudar a implementa��o do protocolo \textit{OSPF}, feita pela \textit{Quagga Routing Suite}, para explorar as causas do intervalo gasto para calcular um novo caminho m�nimo. Em seguida, ap�s a an�lise dos resultados, propor e implementar uma solu��o para diminuir ou remover o tempo gasto para encontrar um novo caminho m�nimo. Com o objetivo de reduzir o intervalo que a rede deixa de enviar pacotes durante a falha de um enlace.

%Os resultados foram obtidos atrav�s de simula��es realizadas na topologia descrita no in�cio do cap�tulo. Para simular rede maiores �  usar geradores de topologia, para n�o precisar configurar cada roteador manualmente. Atualmente existe um m�dulo do NS-3 chamado \textit{Boston university Representative Internet Topology gEnerator} (BRITE) que tem a fun��o de gerar grandes topologias. Entretanto, este m�dulo n�o faz integra��o com o m�dulo DCE. Ou seja, ao utilizar este m�dulo para criar a topologia da rede os roteadores n�o podem executar o protocolo OSPF fornecido pelo DCE.

A Internet � organizada em regi�es chamadas sistemas aut�nomos. Os protocolos respons�veis pelo roteamento dentro dos sistemas aut�nomos s�o chamados de protocolos de roteamento interno. Um destes protocolos utilizado atualmente � o OSPF. Al�m disso, este protocolo pertence a uma classe conhecida como estado de enlace. O OSPF utiliza descri��es do estado dos enlaces da rede para criar uma representa��o da topologia da rede na forma de um grafo. A partir dessa representa��o, os roteadores utilizam o algoritmo de Djikstra para calcular o caminho m�nimo de uma origem para cada destino da rede. A cria��o e manuten��o desta representa��o da topologia � feita por tr�s subprotocolos que juntos formam o OSPF. O protocolo \textit{Hello} tem o objetivo de verificar o estado dos enlaces e eleger um roteador designado e um roteador reserva. Quando dois roteadores estabelecem conectividade bidirecional, o protocolo \textit{Exchange} � repons�vel pela sincroniza��o da representa��o local da topologia da rede. O protocolo \textit{Flooding} � repons�vel por informar a todos os roteadores quando ocorre uma falha na rede.

Neste trabalho descrevemos como simular o OSPF no NS-3. Inicialmente � avaliado o uso do \textit{Global Route Manager} que � respos�vel por preencher as tabelas de roteamento do roteadores. Entretanto, o \textit{Global Route Manager} n�o implementa um protocolo de roteamento. Uma vez que, ele somente calcula caminhos est�ticos, pois as tabelas de roteamento s�o preenchidas antes de iniciar o tr�fego de pacotes na rede. Portanto, n�o s�o utilizados pacotes de controle e quando ocorre uma falha na rede os rotedores n�o s�o informados. Sendo assim, � poss�vel perceber que a implementa��o nativa do NS-3 � muito limitada. Uma vez que, quando um enlace falha, os roteadores da rede n�o encontram novos caminhos. Fica claro que o funcionamento dos subprotocolos \textit{Hello} e \textit{Flooding} n�o s�o simulados. Portanto, um m�dulo do NS-3 chamado DCE passa a ser uma alternativa para simular o protocolo OSPF. O DCE fornece a implementa��o do protocolo OSPF feita pela \textit{Quagga Routing Suite}. O Quagga possui a implementa��o dos tr�s subprotocolos do OSPF. Portanto, os roteadores descobrem quando ocorre uma falha na rede e encontram um novo caminho m�nimo.

Realizar simula��es do OSPF usando a implementa��o feita pelo Quagga apresenta uma s�rie de desafios. Como o DCE n�o � nativo do NS-3 o procedimento de simular o OSPF se torna mais complexo. A maneira de configurar a simula��o passa por diversas mudan�as quando � utilizado o DCE. Deste modo, grande parte do conhecimento adquirido ao utilizar o NS-3, sem o DCE, se torna obsoleto. Ao consultar a documenta��o do DCE para compreender alguns dos procedimentos usados na configura��o das novas simula��es � poss�vel perceber que a documenta��o do DCE, especialmente a parte relacionada ao OSPF, � insuficiente e incompleta. Na verdade, usar o DCE implica em trabalhar com a pr�pria implementa��o do protocolo no sistema operacional, o que elimina a vantagem da simula��o, na medida em que m�ltiplos detalhes devem ser considerados mesmo para testes simples. O pesquisador, em geral, recorre a uma simula��o para obter resultados de forma simples e direta, sem que seja necess�rio analisar os detalhes das implementa��es. Assim, � poss�vel concluir que o trabalho futuro necess�rio, para permitir a continuidade deste trabalho, � implementar o OSPF no NS-3.



%As simula��es foram realizadas com uma topologia simples. Para simular rede complexas � necess�rio utilizar geradores de topologia. Caso contr�rio cada roteador deve ser configurado manualmente. Atualmente existe um m�dulo do NS-3 chamado \textit{Boston university Representative Internet Topology gEnerator} (BRITE) que tem a fun��o de gerar grandes topologias. Entretanto, este m�dulo n�o faz integra��o com o m�dulo DCE. Ou seja, ao utilizar o m�dulo BRITE para criar a topologia da rede os roteadores n�o podem executar o protocolo OSPF fornecido pelo DCE.

%\input{anexo1.tex}     % se houver anexo

\bibliographystyle{brazil}
\bibliography{bibliografia}
% utilize macros (3 primeiras letras do mes em ingles, minusculas) no seu
% .bib para atribuir o nome do mes em portugues nas referencia,
% se o style for brazil, outros estilos tambem aceitam estas macros
% Ex:
%
% @InProceedings{teste,
%   author =       {Luciano}
%   year =         {2013}
%   month =        {}#sep;
% }
%
\addcontentsline{toc}{chapter}{\MakeUppercase{Bibliografia}}

%\singlespacing
%\makecapadissertacao

\end{document}
