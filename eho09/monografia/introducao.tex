\chapter{Introdu\c{c}\~ao}
\label{Introducao}

Robôs móveis são sistemas incorporados no mundo real que se movem autonomamente e interagem com ele para realizar suas tarefas\cite{construcaoMapas2}. 
	A utilização de tais robôs já é frequente hoje em dia e as tarefas a eles designadas estão aumentando em complexidade.
Eles podem ser utilizados em várias aplicações, como limpeza, corte de grama, detectar riscos, explorações de ambientes desconhecidos, 
vigilância autônoma, e assistência a idosos ou pessoas com alguma incapacidade. 
Robôs podem cooperar pra realizar tarefas em comum,
como encontrar e resgatar sobreviventes de um terremoto em território urbano \cite{mobileRobotEnergy}. 

Para o robô se mover de forma autônoma ele deve ser capaz de realizar tarefas como a de planejamento de caminho, navegação, localização, 
evitar obstáculos, controle de motores, construção e atualização de mapas.
	\begin{comment}
	Na navegação do robô é ideal que ele possa mover-se de um ponto inicial, até a posição objetivo, com a capacidade de evitar obstáculos, 
	fazendo o melhor trajeto possível, levando em consideração fatores como a suavidade do trajeto, distância percorrida, energia dispendida e segurança.
	
	O planejamento de movimentos deve levar em consideração fatores como a representação geométrica do ambiente, do modelo de movimento do robô, suavidade do trajeto, 
	comprimento, entre outros. Os principais métodos utilizados na resolução do planejamento de movmentos são: \textit{roadmap}, 
	decomposição de células, e campo potencial. Na maioria deles cria-se uma estrutura especial e aplica-se algoritmos de 
	 busca em grafos como A* \cite{dlite} e Dijkstra\cite{voronoi}.
	\end{comment}
	
	Na navegação do robô é essencial que o robô conheça o ambiente na qual ele realizará sua tarefa. 
Um mapa é uma representação espacial utilizada para registrar a localização de elementos relevantes \cite{construcaoMapas2}.
Tipicamente, existem duas abordagens para a representação de mapas de ambiente \cite{construcaoMapas}:
 mapas métricos e mapas topológicos. Mapas métricos contêm informação da
 geometria do ambiente, da posição dos objetos e distâncias entre esses. Os mapas
topológicos não possuem qualquer informação sobre a geometria do ambiente, à eles são
representados por elos conectados a nós \cite{construcaoMapas}. 

	Na navegação com mapas topológicos, utilizam os nós do mapa para que o robô possa tomar certas ações, como por
exemplo, virar à direita ou à esquerda, e reconhecer marcos para se localizar no ambiente. 
O reconhecimento dos marcos visuais dá ao robô uma localização qualitativa no ambiente,
obtendo sua posição em termos do quanto está mais próximo ou mais distante do alvo. Na
navegação com mapas geométricos a localização é quantitativa, sabendo o robô a sua posição
exata no ambiente \cite{construcaoMapas}.

	Na navegação autônoma, os elementos representados em um mapa podem ser utilizados para diversas finalidades. Por exemplo,
eles podem ser usados para planejar um caminho entre a posição atual do robô e seu
destino \cite{cnn}, especialmente se o mapa representar as áreas por onde ele for permitido navegar. Os mapas podem também ser
utilizados para localizar o robô: comparando os elementos sensoreados no ambiente com aqueles registrados no mapa, o robô pode
inferir o lugar ou possíveis lugares onde está.
	
	Há um conjunto de algoritmos de localização que utilizam uma rede de sensores sem fio (\textit{Wireless sensor networks} - WSNs), 
	na localização do robô, 
	onde o cálculo da localização depende das informações de localização dos nodos (podem ser estáticos ou móveis) da rede\cite{omc}.
	A posição de um certo nodo, pode ser obtida, a partir da posição ou direção dos nodos conhecidos por ele.%, para isso, geralmente 
	%é necessário algum tipo de hardware especial. 
	
	Tempo de chegada (\textit{Time of Arrival} - TOA) \cite{gps}, ângulo de chegada (\textit{Angle of Arrival} - AOA) \cite{aoa}, 
	diferença de chegada de sinais diferentes (\textit{Time Difference of Arrival} - TDOA) \cite{tdoa} 
	e força do sinal recebido (\textit{Received Signal Strength} - RSS) \cite{wifiRadar}, são alguns das abordagens feitas por algoritmos que utilizam WSNs,
	para calculo da localização.
	
	Este trabalho traz uma proposta de localização \textit{indoor} baseado em RSS para robôs móveis,
	utilizando como base o método empírico sugerido no artigo\cite{wifiRadar}. Nesse trabalho
	é utilizada a plataforma Android, pois ela \cite{androidSite} é uma plataforma de fácil desenvolvimento, 
	com documentação farta \cite{androidDev}, e possui \textit{drivers} para câmera, wifi, acelerômetro, 
	\textit{bluetooth}, microfone, compasso e GPS \cite{androidRobot}. 
  
\clearpage
\section{Objetivo}
	Implementar um sistema de localização \textit{indoor} baseado em RSS para robôs móveis,
	desenvolvendo o trabalho de graduação realizado por 
  Alexandre Umezaki e Walter Mazuroski em 2010, no qual implementaram o método empírico proposto
  por Bahl e Padmanabhan no artigo \cite{wifiRadar}.
  
  O sistema implementado por Alexandre Umezaki e Walter Mazuroski é dividido em duas etapas. 
  Na primeira etapa há a coleta de \textit{``fingerprints''} de RSS dos \textit{access points}, onde são armazenadas em
  uma tabela.%, cada registro da tabela,  possui as coordenadas das amostras, 
  %a força do sinal wifi e o nome do AP. 
  
  A segunda etapa, na qual é realizado o processo de localização 
  em si, o sistema recebe a força do sinal wifi de três APs, e calculava a distância euclidiana dos 
  três APs observados em relação as amostras feitas na primeira etapa. 
  A amostra com a menor distância euclidiana é dada como a provável localização do nodo. 
  Esse sistema foi feito para Linux, utilizando a linguagem C
  e as APIs (\textit{Application Programming Interface}) Wireless Tools e o Wireless Extension\cite{apic}
 para coletar as amostras de força do sinal.
  
  O sistema desenvolvido nesse trabalho se diferencia do sistema previamente apresentado em
  especialmente dois aspectos. Primeiro a plataforma adotada: Android. Como mostrado anteriormente,
  o Android possui muitos recursos que podem ser empregados na robótica e a obtenção de informações 
  sobre os \textit{access points} são facilmente obtidos pelo \textit{framework} de aplicativos dessa plataforma como
  pode ser visto em\cite{getRss}.
  
  A segunda diferença está na obtenção do posição do nodo. O sistema aqui implementado, gera um conjunto de 
  pontos um conjunto de pontos que definem uma região no qual o nodo possa estar. A localização desse nodo 
  é estimada através do cálculo do centroide desse conjunto de pontos.
   
\begin{comment}
  Implementar um sistema de localização \textit{indoor} baseado em RSS para robôs móveis, 
  utilizando a plataforma Android. Tendo como base o 
  método empírico proposto por Bahl e Padmanabhan no artigo \cite{wifiRadar}, 
  aplicando o método dos quadrados mínimos para encontrar a estimativa que mais 
  se aproxima da posição real do robô.
  
    O objetivo deste trabalho é implementar um sistema de navegação para robôs móveis em ambientes dinâmicos, utilizando as plataformas Android e Arduino. 
    E no mesmo, apresentar uma solução para os problemas de construção e atualização de mapas, localização e planejamento de caminhos.
    
    Método de representação de mapas utilizado nesse trabalho será similar ao proposto em \cite{cnn}, onde o mapa é um \textit{grid},
    no qual cada célula um valor, que indica o grau de incerteza de haver um obstáculo. O mapa será construído a partir de uma imagem, ela
    será dividida em células, em cada célula será aplicada a função de transformação de Hough\cite{openCV}, e assim será atribuido um valor a célula. A atualização do 
    mapa será feita através das informações coletadas pelo sonar do robô.
    
    Nesse trabalho para fazer a localização do robô será implementado o metodo empírico sugerido no artigo\cite{wifiRadar}, 
	o método é parte de uma técnica de localização baseada em RSS, a qual é uma característica 
	do sinal transmitido, muito utilizada em técnicas de localização por não demandar \textit{hardware} extra.
	
    O planejamento de trajeto do robô será feito aplicando o algoritmo A*\cite{aestrela} no grafo do mapa topológico.
\end{comment}

\section{Organização do trabalho}

  Este trabalho está dividido em 5 partes. 
  A primeira parte contempla as atividades de construção de mapas e localização que um robô móvel deve desempenhar. 
  Ela traz também, um dos grandes desafios da robótica, 
  o problema de auto localização e construção de mapas de ambiente simultâneos(SLAM).
  
  Localização usando rede de sensores sem fio é o tema da segundo parte, onde são mostradas as principais técnicas utilizadas nesse tópico.
  
  A terceira parte aborda a plataforma Android, mostrando a sua arquitetura e as principais característica de uma aplicações Android.
  
  Na quarta parte são mostradas a proposta de localização para robôs móveis utilizada nesse trabalho, como foi implementado e os resultados obtidos.
  
  A ultima parte traz a conclusão desse trabalho e os possíveis trabalhos futuros.

\begin{comment}
  A primeira parte aborda o tópico construção de mapas, apresentando as principais técnicas de construção de mapas e mostra um pouco sobre o problema de auto localização e mapeamento simultâneos.
  
  A segunda parte mostra como pode-se obter a localização de um nodo, utilizando redes de sensores sem fio, e trás um resumo do método de localização a ser utilizado no sistema 
  de navegação.
  
  A terceira parte apresenta o problema planejamento de trajetos, e mostra algumas técnicas utilizadas na solução desse problema.
  
  A quarta parte mostra como será implementado o sistema proposto nesse trabalho, para as plataformas Android e Arduino.
  
  A quinta parte faz a conclusão desse trabalho e apresenta uma breve discussão sobre trabalhos futuros.
\end{comment}