%Texto do resumo....
	Nos últimos anos, tem-se observado um grande crescimento nas pesquisas e utilização de sistemas robóticos no dia-dia.
	E cada vez mais pesquisadores em robótica têm se concentrado no desenvolvimento de robôs móveis, fazendo com que os
	 robôs possam se mover e interagir com o ambiente de forma autônoma, 
	o que abre um vasto campo de novas aplicações e, consequentemente, muitos desafios. Este trabalho 
	apresenta alguns desses desafios como:  localização, construção de mapas e o problema
	de auto localização e construção 
  de mapas de ambiente simultâneos(\textit{Simultaneous Localization and Map Building - SLAM}). 
		
  Este trabalho tem como objetivo implementar um sistema de localização para robôs móveis, 
  baseado no método empírico sugerido no artigo \cite{wifiRadar},
  que prove a localização de um terminal móvel em um ambiente \textit{indoor}. 
  O método é baseado na força do 
  sinal recebido(\textit{Received Signal Strength} - RSS), que é muito explorada em
  em técnicas de localização em uma rede de sensores sem fio pela fácil aplicabilidade. Serão apresentadas também, outras 
  técnicas utilizadas na localização em uma rede de sensores sem fio.
  
	Todo o sistema foi desenvolvido na plataforma Android, pela sua fácil incorporação a robótica devido ao 
	grande quantidade de sensores suportados. Serão brevemente apresentadas
	algumas características deste sistema operacional.
   
\textbf{Palavras-chave:} RSS, Localização, Robótica, Android, WSNs.