\documentclass[12pt]{article}

\usepackage{sbc-template}
\usepackage{mathtools}
\usepackage{graphicx,url}
\usepackage[brazil]{babel}
 
\usepackage[utf8]{inputenc}  

     
\sloppy

\title{Relatório}

\author{Edileuton Henrique de Oliveira - GRR20091930\\{\em eho09@inf.ufpr.br}  }

\address{Universidade Federal do Paraná(UFPR)}
  
\begin{document} 

\maketitle

\begin{resumo} 

	Este relatório apresentará uma proposta de um sistema, para navegação de robôs móveis em ambientes dinâmicos. Ele será dividido em duas aplicações,
utilizando as plataformas Android\cite{androidSite}-Arduino\cite{arduinoSite}. A aplicação que rodará no robô baseado no Arduino fará a comunicação
com os sensores e motores do robô, e repassará as informações coletadas para aplicação Android via bluetooth. A aplicação que rodará no dispositivo Android
será responsável por tomar decisões a partir dos dados coletados, pelo robô e pelos sensores do tablet, e assim determinar a localização do robô, e realizar a
construção e atualização do mapa do ambiente, e o planejamento do trajeto a ser utilizado.
	

\end{resumo}

\section{Introdução}
    
    %MOBILE robots can be used in many applications, such as carpet cleaning, lawn mowing, hazard detection, and
%exploration of unknown areas. Multiple robots may work collectively to accomplish a common task, for example, searching
%and rescuing survivors in an urban area after an earthquake. Ex-
%isting studies about mobile robots focus mostly on enhancing
%individual robots’ capability, such as sensing, obstacle detec-
%tion and avoidance, localization, motion planning, or interac-
%tions with human controllers.
	Robôs móveis são sistemas incorporados no mundo real que se movem autonomamente e interagem com ele para realizar suas tarefas\cite{construcaoMapas2}. 
	A utilização de tais robôs são, até certo ponto, freqüentes hoje em dia, e as tarefas a eles designadas estão aumentando em complexidade.
Eles podem ser utilizados em várias aplicações, como limpeza, cortar grama, detectar riscos, explorações de ambientes desconhecidos, 
vigilância autônoma, e assistência a idosos ou pessoas incapazes. 
Robôs podem cooperar pra realizar tarefas em comum,
como encontrar e resgatar sobreviventes de um terremoto em território urbano \cite{mobileRobotEnergy}. 

	Este trabalho apresenta uma proposta para a navegação de um robô móvel em ambientes dinâmicos. A navegação de um robô móvel corresponde à tarefa
de autonomamente mover-se de uma posição específica, ou região, para uma outra posição ou região \cite{construcaoMapas2}. Esta tarefa pode envolver simplesmente o 
movimento do quarto das crianças para a cozinha, em uma casa, o movimento entre um prédio e outro no
campus da universidade, ou até uma viagem entre uma cidade e outra \cite{construcaoMapas2}.
	
	Dentro do problema navegação em robótica, podemos destacar a construção de mapas, pois, para navegação do robô, 
é essencial que o robô conheça o ambiente na qual ele realizará sua tarefa. 
Um mapa é uma representação espacial utilizada para descobrir a localização de elementos especiais \cite{construcaoMapas2}.
Tipicamente, existem duas abordagens para a representação de mapas de ambiente \cite{construcaoMapas}:
 mapas métricos e mapas topológicos. Mapas métricos contêm informação da
 geometria do ambiente, da posição dos objetos e distâncias entre esses. Os mapas
topológicos não possuem qualquer informação sobre a geometria do ambiente, eles são
representados por elos conectados a nós \cite{construcaoMapas}. 

	Nos nós o robô pode tomar certas ações, como por
exemplo, virar à direita ou à esquerda, e reconhecer marcos para se localizar no ambiente. 
O reconhecimento dos marcos visuais dá ao robô uma localização qualitativa no ambiente,
obtendo sua posição em termos do quanto está mais próximo ou mais distante do alvo. Na
navegação com mapas geométricos a localização é quantitativa, sabendo o robô a sua posição
exata no ambiente \cite{construcaoMapas}.

	Na navegação autônoma, os elementos representados em um mapa podem ser utilizados para diversas finalidades. Por exemplo,
eles podem ser usados para planejar um caminho entre a posição atual do robô e seu
destino \cite{cnn}, especialmente se o mapa representar as áreas por onde ele for permitido navegar. Os mapas podem também ser
utilizados para localizar o robô: comparando os elementos sensoriados no ambiente com aqueles registrados no mapa, o robô pode
inferir o lugar ou possíveis lugares onde está.
	
	Há um conjunto de algoritmos de localização que utilizam uma rede de sensores sem-fio (\textit{'Wireless sensor networks'} - WSNs). Onde o 
	calculo da localização depende das informações de localização dos nodos(que podem ser estáticos ou móveis) da rede\cite{omc}.
	A posição de um certo nodo, pode ser obtida, a partir da posição ou direção dos nodos conhecidos por ele, para isso, geralmente 
	é necessário algum tipo de hardware especial. 
	
	Tempo de chegada (\textit{'Time of Arrival'} - TOA) \cite{gps}, ângulo de chegada (\textit{'Angle of Arrival'} - AOA) \cite{aoa}, 
	diferença de chegada de dois sinais diferentes (\textit{'Time Difference of Arrival of Two Different Signals'} - TDOA) \cite{tdoa} 
	e força do sinal recebido (\textit{'Received Signal Strength'} - RSS) \cite{wifiRadar}, são alguns das abordagens feitas por algoritmos que utilizam WSNs,
	para calculo da localização.	
	
	Na navegação do robô é ideal que ele possa mover-se de um ponto inicial, até a posição objetivo, com a capacidade de evitar obstáculos, 
	fazendo o melhor trajeto possível, levando em consideração a suavidade do trajeto, distância percorrida, etc. Por isso, planejamento de trajetos, 
	vem sendo extensivamente estudado, e é um dos principais problemas dentro da robótica.
	
	Existem um grande número de métodos para resolver o planejamento de movimentos. Entretanto, nem todas generalizações,
	deste problema podem ser aplicadas. Por exemplo, alguns métodos necessitam que espaço de trabalho seja bidimensional e os obstáculos poligonais.
	Apesar de muitas diferenças externas, os métodos são baseados em algumas técnicas gerais: \textit{roadmap}, decomposição de células,
	 e campo potencial. Na maioria deles cria-se uma estrutura especial e aplica-se algoritmos de 
	 busca em grafos como A*\cite{dlite} e Dijkstra\cite{voronoi}.
	
	
	%Android - Arduino
	Como sabemos o Android OS \cite{androidSite} é uma plataforma de fácil desenvolvimento, com documentação farta \cite{androidDev}, e possui drivers para câmera, wifi, acelerômetro, 
	bluetooth, microfone e GPS, o que torna o desenvolvimento para robôs menos custoso\cite{androidRobot}, por isso, da sua utilização.  Do sistema de navegação, 
	a aplicação que rodará no Android será o cérebro do sistema, ela será responsável por tomar decisões analisando as informações disponíveis.
  Esta aplicação coletará a força de sinal wifi dos \textit{Access Points}, para determinar a posição do robô no mapa. Ela deverá ser capaz de
   planejar o trajeto a ser realizado pelo robô, e repassar os comandos de movimentação para este, a partir do mapa topológico e das 
  informações obtidas do robô.
  
  O robô utilizado é baseado na plataforma Arduino. 
  O Arduino é uma plataforma \textit{open-source} de prototipagem eletrônica baseado no flexible, e de fácil uso de \textit{hardware} e \textit{software} \cite{arduinoSite}.
  Ele pode monitorar o ambiente, recebendo entradas de uma grande variedade de sensores, e pode interagir com o ambiente controlando luzes, motores, etc. O microcontrolador
  integrado, pode ser programado utilizando a \textit{Arduino programming language} e a \textit{Arduino development environment}. Ela será utilizada, pois, suas
   ferramentas são acessíveis, com baixo custo, flexíveis e fáceis de se usar. A aplicação que será executado nessa plataforma será responsável por interagir com 
   a aplicação do Android, executar os comandos recebidos. E também repassar informações, como dados do sonar e giro dos motores.
  
 
 \subsection{Objetivo}
    
    O objetivo deste trabalho é implementar um sistema de navegação para robôs móveis em ambientes dinâmicos, utilizando as plataformas Android e Arduino. 
    E no mesmo, apresentar uma solução para os problemas de construção e atualização de mapas, localização e planejamento de caminhos.
    
    Método de representação de mapas utilizado nesse trabalho será similar ao proposto em \cite{cnn}, onde o mapa é um \textit{grid},
    no qual cada célula um valor, que indica o grau de incerteza de haver um obstáculo. O mapa será construído a partir de uma imagem, ela
    será dividida em células, em cada célula será aplicada a função de transformação de Hough\cite{openCV}, e assim será atribuido um valor a célula. A atualização do 
    mapa será feita através das informações coletadas pelo sonar do robô.
    
    Nesse trabalho para fazer a localização do robô será implementado o metodo empírico sugerido no artigo\cite{wifiRadar}, 
	o método é parte de uma técnica de localização baseada em RSS, a qual é uma característica 
	do sinal transmitido, muito utilizada em técnicas de localização por não demandar \textit{hardware} extra.
	
    O planejamento de trajeto do robô será feito aplicando o algoritmo A*\cite{aestrela} no grafo do mapa topológico.
    
 \subsection{Organização do trabalho}
  A primeira parte aborda o tópico construção de mapas, apresentando as principais técnicas de construção de mapas e mostra um pouco sobre o problema de auto localização e mapeamento simultâneos.
  
  A segunda parte mostra como pode-se obter a localização de um nodo, utilizando redes de sensores sem fio, e trás um resumo do método de localização a ser utilizado no sistema 
  de navegação.
  
  A terceira parte apresenta o problema planejamento de trajetos, e mostra algumas técnicas utilizadas na solução desse problema.
  
  A quarta parte mostra como será implementado o sistema proposto nesse trabalho, para as plataformas Android e Arduino.
  
  A quinta parte faz a conclusão desse trabalho e apresenta uma breve discussão sobre trabalhos futuros.
 
\clearpage
\section{Construção de Mapas}
A tarefa de mapeamento corresponde à atribuição de valores aos elementos do mapa, relacionando cada um a uma certa 
posição nele \cite{construcaoMapas2}. O tipo ideal de mapa
a ser utilizado ou construído por um robô móvel depende da tarefa e do ambiente onde
este esteja inserido. Também depende das características do robô, tais como os tipos
de sensores que ele tem, bem como a forma com ele se move\cite{construcaoMapas2}.  Há duas abordagens principais para a 
representação de mapas de ambiente \cite{construcaoMapas}.
São os mapas métricos e topológicos, que serão discutidos a seguir. 

\subsection{Construção de Mapas Métricos}
No caso da construção de mapas métricos, o objetivo é obter um mapa detalhado do ambiente, 
com informações sobre a forma e o tamanho dos objetos e os limites das
áreas livres para a navegação, tais como corredores, quartos, trilhas e estradas. Para
representar essa complexa e detalhada informação, o mapa é geralmente dividido em
uma densa rede de forma que cada célula contenha informações sobre a ocupação
desse espaço por um objeto e, possivelmente, outras características ambientais que
estão sendo mapeadas \cite{construcaoMapas2}. 

Como o mapa métrico salva informações no próprio domínio do sinal do sensor, a
tarefa de localização é executada de uma forma simples. Além disso, devido à alta
densidade de informação nesses mapas, o resultado da localização é mais preciso e
menos sujeito a ambigüidades \cite{construcaoMapas2}. Como os mapas métricos 
demandam uma grande quantidade de informações, eles exigem muita capacidade de processamento e 
armazenamento.

Atualmente grande parte dos sistemas de mapeamento assumem que o ambiente é estático
durante o mapeamento. Se uma pessoa anda dentro do alcance dos sensores do robô durante o
mapeamento, o mapa resultante conterá evidências a respeito de um objeto na localização
correspondente. Além disso, se o robô retornar para esta localização e varrer a área uma
segunda vez, a estimativa da posição será menos precisa, uma vez que as novas medições não
contêm nenhum vestígio correspondente àquela pessoa. A exatidão reduzida do mapa
resultante pode ter uma influência negativa no desempenho do robô.

\subsection{Construção de Topológicos}
No caso de mapas topológicos, o objetivo é construir uma estrutura relacional, normalmente um grafo, 
de forma geral, registram informações sobre determinados
elementos ou locais do ambiente, chamados marcos \cite{construcaoMapas2}. Assim, marcos e relações são
os elementos dos mapas topológicos.
Essas relações podem ser de vários tipos. Eles podem indicar o deslocamento
relativo entre dois marcos, a existência de um caminho entre eles, etc. É fácil perceber que as informações contidas no mapa topológico é
menos detalhado e distribui-se de forma dispersa, concentrando-se apenas em alguns pontos de interesse.
Assim, é muito seletivo com relação à informação que ele registra \cite{construcaoMapas2}. 

O mapa topológico também demanda que o processamento
dos dados dos sensores do robô, sejam feitos de forma mais abstrata, a fim de extrair informações sobre as localidades mais relevantes que devem ser consideradas
como marcos.

\subsection{Problema de auto localização e mapeamento simultâneos}
  O problema de auto localização e construção de mapas de ambiente simultâneos (\textit{'Simultaneous Localization and
Map Building' - SLAM}) consiste em um robô autônomo iniciar a navegação
em uma localização desconhecida, em um ambiente desconhecido e então construir um mapa
desse ambiente de maneira incremental, enquanto utiliza o mapa simultaneamente para calcular a sua localização \cite{slam}.
 O SLAM tem cido tema de várias pesquisas na área de robótica, a grande vantagem do SLAM é que elimina a necessidade de estruturas artificiais ou
um conhecimento topológico a priori do ambiente. Há 3 abordagens principais que são utilizadas no problema do SLAM. 

O primeiro e mais popular deles é baseado na teoria da estimação usando filtro de Kalman \cite{slam}. ele
fornece uma solução recursiva para o problema da navegação e uma maneira de calcular
estimativas consistentes para a incerteza na localização do robô e nas posições dos marcos do
ambiente, com base em modelos estatísticos para o movimento dos robôs e observações
relativas dos marcos do ambiente\cite{slam}.

O segundo método consiste em evitar a necessidade de estimativas absolutas da
posição e de medições precisas das incertezas para utilizar conhecimento mais qualitativo da
localização relativa dos marcos do ambiente e do robô para a construção do mapa global e
planejamento da trajetória\cite{construcaoMapas}.

O terceiro método é bastante amplo e afasta-se do rigor matemático do filtro de
Kalman, ou do formalismo estatístico, retendo uma abordagem essencialmente numérica ou
computacional para a navegação e para o problema de SLAM. Essa abordagem inclui o uso
de correspondência com marcos artificiais do ambiente\cite{construcaoMapas}.

\subsection{Problema de navegação e mapeamento completo do ambiente}
  No artigo \cite{cnn} - \textit{"A Bioinspired Neural Network for Real-Time Concurrent Map Building and Complete Coverage Robot Navigation in Unknown Environments"},
Luo e Yang propõem um modelo de mapeamento e navegação de robôs em ambientes desconhecidos e dinâmicos. Eles utilizam robôs limpadores para exemplificar o modelo. No 
qual o robô deve percorrer todo o local, para fazer sua limpeza.

  O mapa topológico do ambiente é discretizado em células, onde as células podem assumir o formato de quadrados, retângulos e triângulos. 
Utilizando quadrados e retângulos, o robô limpador têm 8 direções possíveis para deslocamento, enquanto que utilizando células triangulares, 
o robô possui 12 direções possíveis, assim permitindo que o robô navegue por caminhos mais curtos e flexíveis. 
Essas células formam uma rede neural, e cada célula pode assumir um estado: desconhecido, limpo, sujo e \textit{deadlock}. 

O mapeamento do ambiente é feito com o robô limpador percorrendo o ambiente em "ziguezague" e monitorando a atividade neural da rede.
A dinamicidade da atividade neural, representa as mudanças do ambiente. As areas sujas e com com obstáculos ficam no topo e no vale da atividade da rede neural, 
respectivamente. As áreas sujas atraem o robô, através da propagação da atividade neural. 
O planejamento do caminho para evitar colisões é feito em tempo real, baseado na dinamicidade da atividade neural da rede e na posição anterior do robô, assim ele 
é capaz de percorrer o ambiente, limpando as áreas sujas.
 
 \clearpage
\section{Localização utilizando WSNs}
	WSNs é uma rede de sensores que coleta informações em um campo monitorado. Este tipo de rede tem sido utilizada em 
	várias aplicações, incluindo rastreamento de objetos, resgate, monitoramento de ambiente, entre outros \cite{omc}. Localização é um 
	tópico muito importante, pois, muitas aplicações das WSNs dependem do conhecimento das posições dos sensores. Na maioria das WSNs 
	os sensores são estáticos, enquanto em algumas aplicações modernas o sensores devem ser móveis.
	
	Há um conjunto de algoritmos de localização que utilizam WSNs. Onde o calculo da localização depende das informações 
	de localização dos nodos(que podem ser estáticos ou móveis) da rede. Esses algoritmos podem ser divididos em duas categorias: \textit{range-based}
	e \textit{range-free}. Algoritmos do tipo \textit{range-free} \cite{omc} \cite{wsnsLinear}, geralmente, requerem que os nodos conhecidos, estejam dentro 
	do raio de comunicação dos sensores em comum, eles tem sido muito explorados pois não precisam de \textit{hardware} extra. Os algoritmos do tipo \textit{range-based},
	geralmente necessitam de algum tipo de \textit{hardware} especial, onde a posição de um certo nodo, pode ser obtida, a partir da posição ou direção dos nodos conhecidos por ele. Este trabalho 
	se foca em mostra algoritmos do tipo \textit{range-based}.
	
\subsection{Localização baseada em TOA}
	TOA é o tempo medido em que um sinal (rádio frequência, acústicos, ou outros) chega a um
receptor pela primeira vez \cite{gps}. O valor medido é o tempo de transmissão somado ao
atraso do tempo de propagação. Este atraso,$T_{i,j}$
, entre a transmissão do sensor i e recepção do sensor j, é igual a distância entre o transmissor e o receptor, 
$d_{i,j}$, dividido pela velocidade de propagação do sinal, $v_{p}$
. Esta velocidade para a rádio frequência é aproximadamente $10^{6}$  vezes mais rápida que a velocidade do som \cite{gps},
ou seja, 1 ms corresponde a 0,3 m na propagação sonora, enquanto para a rádio frequência, 1 ns corresponde a 0,3 m \cite{gps}.

O ponto chave das técnicas baseadas em tempo de a capacidade do receptor de
estimar com precisão o tempo de chegada em sinais na linha da visão. Esta estimativa é prejudicada tanto pelo ruído 
aditivo quanto por sinais de multipercurso.

\subsection{Localização baseada em AOA}
      Existem duas maneiras nas quais sensores medem o AOA \cite{aoa}. O método mais
comum é usar um vetor de sensores e utilizar a matriz de processamento de sinal nos nós sensores. Neste caso, cada nó sensor é
composto de dois ou mais sensores individuais (microfones para sinais acústicos ou antenas de sinais de rádio frequência),
cujas posições com relação ao centro do nó são conhecidos. A AOA é estimada a partir
das diferenças de tempos de chegada de uma transmissão do sinal em cada um dos elementos da matriz de sensores.

  A segunda abordagem para a estimativa do AOA, usa a razão RSS entre duas (ou
mais) antenas direcionais localizadas no sensor. Duas antenas direcionais apontadas em
direções diferentes, de tal maneira que suas hastes se sobrepõem, podem ser usadas para
estimar a AOA partir da relação entre seus valores individuais RSS.

\subsection{Localização baseada em TDOA}
  Os algoritmos de TDOA fazem a medição da diferença no tempo de recepção
de sinais de diferentes estações de base ('Base Station's' - BSs)), sem a necessidade de uma sincronização de
todos os participantes BSs e terminais móveis(MT)) \cite{tdoa}. Na verdade, a incerteza entre os tempos de referência
dos BSs e do MT pode ser removidas por meio de um cálculo diferencial. Por isso, apenas
os BSs envolvidos no processo de estimativa de localização deve ser bem sincronizados.

Para um par de BSs, digamos i e j, o TDOA,
$T_{ij}$, é dada por $T_{ij} TBSi - TBSj$, onde $TBSi$ e $TBSj$ são os tempos absolutos tomados 
pelo estouro de chegada nos BSs i e j, respectivamente. Supondo que o MT está em LOS (\textit{line-of-sight}) com ambos os BSs, i
e j, o MT deve situar-se em uma hipérbole. Uma segunda hipérbole, onde o
MT deve estar, pode ser obtido através de uma medição adicional de TDOA envolvendo
um terceiro BS. A posição do usuário pode ser identificada como o ponto de intersecção
das duas hipérboles. A solução do sistema de equações pode ser encontrado com um
método iterativo e minimização dos quadrados mínimos \cite{tdoa}.

\subsection{Localização baseada em RSS}	
	A localização baseada em RSS apresenta-se como uma das únicas a não necessitar de hardware extra, e ser uma técnica
de fácil aplicação. E uma técnica duas vezes mais econômica que as demais técnicas conhecidas, 
tanto em termos de equipamento quanto em baixo consumo de energia\cite{rss1}.

  O funcionamento da RSS é baseado na medição do sinal no receptor e o valor
obtido indica a distância até o transmissor \cite{rss1}. Tal forma de medição por muitas vezes é
descartada, pois não leva em conta o ambiente onde está sendo aplicada a medição; logo,
corresponde a um resultado não confiável, obstruções e obstáculos proporcionam erros de grande relevância, 
sendo esse um dos maiores problemas dos métodos de localização baseados em RSS\cite{wifiRadar}. 

\subsection{Proposta de localização utilizando RSS de \textit{Access Points} Wifi}	

   Em \cite{wifiRadar} - \textit{"Radar: an in-building RF-based user location and tracking system"}, Bahl e Padmanabhan propõem o método empírico,
   para localização baseada em rss dentro de um prédio. O método consiste em realizar uma série de medições da força de sinal das estações bases, e a partir dessas medições, 
   a localização pode ser determinada fazendo uma triangulação dos dados coletados. 

  No artigo \cite{wifiRadar}, para obter a posição de um nodo, são medidas as forças do sinal wifi de 3 \textit{Access Points} (rss1, rss2, rss3), 
utiliza-se a distância Euclidiana, para fazer a comparação com os dados previamente coletados $sqrt((rss1-rss1')^{2}+(rss2-rss2')^{2}+(rss3-rss3')^{2})$, 
e assim encontrar o ponto que mais se aproxima dos parâmetros coletados naquele local. 

\clearpage
\section{Planejamento de trajetos}
  O problema planejamento de trajetos, consiste em encontrar o melhor trajeto possível, 
  de um ponto de origem até um ponto de destino que não resulte em colisão com obstáculos \cite{voronoi}.
  Dependendo da quantidade de informação disponível sobre o ambiente, que pode ser completamente ou
   parcialmente conhecido ou desconhecido, as abordagens de planejamento podem variar consideravelmente.
   A definição de melhor trajeto também pode variar. Esse tópico vem ganhando mais relevância em áreas, além da robótica, como computação gráfica, sistemas de informações 
  geográficas e jogos \cite{planejamentoCaminhos}.
  
  A computação geométrica tem um papel especial dentro do desenvolvimento do planejamento de trajetos.
  As principais abordagens utilizando geometria computacional são: \textit{roadmap}, decomposição de células,
 e campo potencial. Elas serão apresentadas a seguir.

\subsection{\textit{Roadmap}}
      A técnica de \textit{roadmap} para planejamento de trajeto consiste em capturar a conectividade
      do espaço livre do ambiente em uma rede de curvas de uma dimensão, denominada \textit{roadmap} \cite{planejamentoTrajetorias}. 
      Uma vez que a rede é obtida, ela é vista como um conjunto de caminhos. O planejamento de trajeto pode ser feito 
      conectando as posições iniciais e finais do robô ao \textit{roadmap} e buscar neste um caminho entre estes 
      dois pontos.
      
      Vários métodos foram propostos, formando \textit{roadmaps} a partir de estruturas de computação 
      geométrica como diagrama de Voronoi e grafo de visibilidade.
      
\subsubsection{Diagrama de Voronoi}
Um diagrama de Voronoi é um estrutura geométrica que
representa informações de proximidade sobre uma série de
pontos ou objetos \cite{tecnicasNavegacao}. Dada uma série de sites ou objetos, o plano
bidimensional em que o robô se locomove geralmente contém
obstáculos, cada um destes obstáculos pode ser representado
por polígonos côncavos ou convexos. Para encontrar o
diagrama generalizado de Voronoi para esta coleção de
polígonos, podemos calcular o diagrama através de uma
aproximação, convertendo os obstáculos em uma série de
pontos \cite{voronoi}. 

  Primeiramente as faces dos polígonos são
subdivididas em uma série de pontos. O próximo passo é
calcular o diagrama de Voronoi para esta coleção de pontos.
Após o diagrama de Voronoi ser calculado, os segmentos do
diagrama que intersectam algum obstáculo são eliminados. E então 
utiliza-se uma algoritmo de busca em grafos para encontrar o melhor caminho.
Este método gera uma rota que na sua maior parte
permanece eqüidistante dos obstáculos, criando um
caminho seguro para o robô se locomover, apesar de não ser o
caminho mais curto. 
  O caminho encontrado pode ser melhorado como é mostrado em \cite{voronoi}, no qual pode-se 
  deixa-lo mais suave, retirando desvios desnecessários.
  
\subsubsection{Grafo de visibilidade}
  Um grafo de visibilidade é obtido gerando-se segmentos de
reta entre os pares de vértices dos obstáculos \cite{tecnicasNavegacao}. Todo o
segmento de reta que estiver inteiramente na região do espaço
livre é adicionada ao grafo. Para executar o
planejamento de trajetória, a posição atual e o objetivo são
tratados como vértices, isso gera um grafo de conectividade
onde utiliza-se algoritmos de procura para se encontrar um
caminho livre. O caminho mais curto que for encontrado no grafo de
visibilidade é o caminho ótimo para o problema especificado\cite{tecnicasNavegacao}.
Mas os caminhos encontrados no grafo podem tocar obstáculos.
Para utilizar um método de navegação com grafo de visibilidade é
necessário quer o mapa seja completo e bem definido.

\subsection{Decomposição de células}
  O método de decomposição de células consiste em dividir o espaço livre do robô em regiôes simples (células) \cite{planejamentoTrajetorias},
  de forma qur um caminho entre quaisquer duas configurações em uma mesma célula possa ser obtido. Um grafo
   não dirigido representando a relação de adjacência entre as células é construído. Os vértice que compôem este grafo são as células extraídas do espaço livre do robô. Há uma 
   aresta entre dois vértices se e somente se as células correspondentes a ele são adjacentes. Uma busca é feita nesse grafo 
   e seu resultado é uma sequência de células denominada canal. Um caminho continuo pode ser retirado do canal. A qualidade de caminho obtido, 
   é diretamente afetado pelo grau de decomposição de células.
  
\subsection{Campo potencial}
   A idéia por trás do método de campo potencial, é assinalar uma função similar ao potencial eletrostático
   para cada obstáculo, e então criar uma estrutura topológica de espaço livre do robô na forma de vales de potencial 
   mínimo \cite{voronoi}. O robô move-se em direção a objetivo, já que este um, gera um campo potencial que atrai o robô. Os obstáculos geram
    uma força de repulsão, que impede que o robô colida com obstáculos. Assim a direção do movimento do 
    robô é determinada pela força proveniente do campo potencial nesta determinada configuração. Um problema desse método, é que no caminho utilizado, o robô 
    pode ficar preso e nunca atingir o objetivo.

\clearpage
\section{Implementação}	 
  Os experimentos serão realizados no departamento de informática da Universidade Federal do Paraná, 
  devido a grande quantidade de Access Points disponíveis e a disponibilidade da planta do departamento. 
  A implementação do sistema de navegação será dividida em duas etapas:
  \subsection{Etapa 1}
  
  Uma aplicação escrita em C será utilizada para criação do mapa topológico do departamento. Essa aplicação recebe como entrada:
  \begin{itemize}
    \item Uma imagem contendo o mapa.
    \item Tamanho da célula.
  \end{itemize}
 
  A imagem será dividida em células. Em todas as células, serão aplicadas a função de transformação de Hough \cite{openCV}, da biblioteca OpenCV para detecção de linhas. 
  Esta aplicação devolve uma matriz, onde célula possui uma valor que indica o grau de incerteza de haver um obstáculo: 0 (sem obstáculo) ou 4(com obstaculo). 
  Essa matriz será embarcada na aplicação Android da etapa 2.
  
  O tamanho ideal da célula, será definido via experimentos. Em \cite{cnn} o tamanho da célula tem o tamanho do robô. O artigo \cite{dlite} sugere que o tamanho da célula deve 
  comportar todo o obstáculo, lembrando que a quantidade de células pode influênciar no tempo de processamento do planejamento da trajetória do robô \cite{voronoi}.
  
    Uma segunda aplicação Android, será utilizada para coletar amostras, como descrito no método empírico do artigo\cite{wifiRadar}. As tabelas resultantes dessa aplicaçao, também serão 
  embarcada na aplicação Android da etapa 2. As linha da tabela conterão as coordenadas, RSS e nome do AP da amostra daquele local. Vale ressaltar, que número de amostras tem grande impácto 
  na precisão do calculo da localização \cite{wifiRadar}.
  
  \subsection{Etapa 2}
	A aplicação Android dessa etapa, possui 2 componentes principais: uma \textit{Acitivity} \cite{activity} que fará a interação com o usuário, 
	e um Service \cite{service} que fará o processo de navegação do robô. O robô e o tablet, se comunicarão via \textit{bluetooth},
	ou seja, para o funcionamento do sistema, o tablet deve estar pareado com o robô.

	A \textit{Acitivity} irá obter a posição do clique em relação à imagem do mapa, e obter as coordenadas da posição de destino
	 do robô e repassar para o Service. O usuário deve clicar no botão conectar para que a aplicação possa iniciar uma conexão \textit{bluetooth} com o robô.
	Haverá uma imagem com o mapa do ambiente, o usuário deve apontar o local no qual o robô deve atingir. E então clicar no botão iniciar.
	
	Um Service será iniciado, e executará os seguintes passos:
	\begin{itemize}
	  \item O mapa topológico será carregado na memória.
	  \item A posição do robô será calculada, medindo o RSS\cite{wifiRss} dos APs e aplicando o método descrito em \cite{wifiRadar}.
	  \item Em seguida será aplicado o algoritmo A*\cite{aestrela} na matriz gerada na etapa 1. A função h (heurística) levará em consideração o valor da 
	célula(celulas com valor inferior a 3 serão descartados do caminho) e a sua distância euclidiana, em relação a célula destino. 
	  \item Com o trajeto definido, um comando(vetor de bytes indicando distância e direção) será enviado ao robô.
	  \item O Service aguarda o recebimento dos dados do robô.
	  \item Se nenhum obstáculo é encontrado um novo comando é enviado, e processo se repete, até que o robô atinja seu objetivo.
	  \item Se o robô encontrar um novo obstáculo, a célula onde este foi encontrado é incrementado em 1. 
	  Então o processo de descoberta de trajeto é iniciado novamente.
	  \item Se por acaso o robô verificar está célula de novo e não houver obstáculo, o valor da célula é decrementado em 1.
	  \item Se o robô ficar preso, então é adotado um procedimento descrito em \cite{dlite},
	  muros virtuais são criado, para que o robô evite este caminho em uma segunda passagem.
	\end{itemize}
	
	 Aplicação Arduino faz a interface com os motores, o sonar e o adaptator \textit{bluetooth}.
	 Basicamente, ela executa o comando recebido, e então envia os dados coletados pelo sonar e aguarda um novo comando.
 
\clearpage
\section{Trabalhos Futuros}
    As possibilidades de trabalhos futuros são enormes. Primeiro poderia implantar o modelo de propagação
de sinal, proposto no artigo \cite{wifiRadar}. Nele são levados em consideração
diversas variáveis para sua elaboração, sendo que a mais importante delas é a quantidade de obstáculos que estão 
entre o transmissor de sinal (Access Point) e o receptor (terminal móvel).
Com base nesse parâmetro, uma equação da distância em função da força de sinal recebido
poderá ser deduzida, que será útil para os cálculos da posição e rastreamento
do terminal móvel. Ou ainda combinar técnicas baseadas em RSS com as tecnologias já largamente empregadas, tais como o GPS.
    
    
    Poderia também, ao contrário do sistema proposto, tratar a possibilidade de não haver um mapa do ambiente, e utilizar técnicas que lidam com o SLAM, 
    como nos artigos \cite{construcaoMapas2}\cite{construcaoMapas}\cite{slam}. E utilizar um algoritmo de planejamento de trajetos mais robusto 
    como o proposto em \cite{voronoi}, que utiliza diagrama de Voronoi para criar um \textit{roadmap}. E ainda ao invés de usar um simples sonar para detectar
     obstáculos, utilizar a câmera do tablet.
     
     Podemos aumentar a escala e ao invés de utilizar um simples robô de 40 cm dentro de um prédio, utilizar o sistema de navegação em um carro, 
      em um ambiente muito maior como pode ser visto em \cite{googleCar}.
\section{Conclusão}
  Os robôs vem sendo muito utilizados na automatização de tarefas, e nesse trabalho podemos perceber que  uma tarefa simples, como deslocar um robô de um lugar a outro, 
  de forma autônoma, exige técnicas complexas de estatística e de várias áreas da computação como inteligência artificial, geometria computacional e processamento de 
  imagens.
  
    O conceito por trás desse trabalho é muito promissor, pois, a idéia de poder clicar em um mapa, ou falar o endereço onde você deseja ir,
    e seu celular ou tablet dirige seu carro para você, sem precisar tocar no volante, é extremamente interessante.
	
\clearpage
\bibliographystyle{IEEEtran}
\bibliography{sbc-template}

\end{document}
