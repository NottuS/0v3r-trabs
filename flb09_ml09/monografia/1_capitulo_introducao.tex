\chapter{Introdu\c{c}\~ao}
\label{Introducao}

% Atualmente a sociedade possui uma ligação muito forte com a internet. A cada segundo muitas pessoas acessam a rede mundial de computadores, e com essa quantidade grande de acessos as plataformas webs necessitam suportar esses acessos simultâneos sem ter problemas, pois qualquer entropia\footnote{Segundo a teoria geral dos sistemas, entropia é consequência da falta de relacionamento entre as partes de um sistema, o que provoca perdas e desperdícios} pode ser considerado um prejuízo para quem depende desta tecnologia. Por isso quando alguém necessitar criar uma aplicação web para suportar múltiplos acessos simultâneos, esta pessoa decidirá pela melhor aplicação escalar para seu problema, uma solução possível é através de um conjunto de ferramentas conhecidas como MEAN que serão abordadas nesta monografia.
Uma aplicação web pode ser resumida em requisições de um cliente ao servidor. Após o servidor ter recebido a requisição, ele envia uma resposta ao cliente. Esta resposta pode ser desde uma página HTML até uma página de erro.

Considere a seguinte situação, muitos clientes requisitando uma página HTML ao mesmo tempo e o servidor tendo que responder a todos eles. Se o servidor tratar esses pedidos como uma fila e somente responder a próxima requisição quando a anterior estiver terminada, os clientes que tiveram suas requisições no fim desta fila, terão maior tempo de resposta implicando em demora para carregar a página do lado do cliente.

Para diminuir o tempo de duração de múltiplas requisições de clientes ao servidor, existem diversas soluções que utilizam técnicas de escalabilidade, alguns exemplos são melhorar o hardware do servidor, com a substituição do hardware antigo por um mais moderno com um poder de processamento maior, e no caso do software, é checar se as ferramentas que estão sendo utilizadas possuem sinergia e foram desenvolvidas para serem escaláveis.

A proposta deste texto é apresentar uma solução via software que visa ajudar a resolver o problema de escalabilidade. O MEAN \textit{Stack}. O MEAN é um acrônimo das quatro tecnologias que são o banco de dados MongoDB, o Node.js e o Express.js na parte do servidor e por último o AngularJS no lado do cliente.

\section{Objetivo}

Nesta monografia serão apresentados conceitos e funcionalidades de um conjunto de tecnologias com o foco em resolver os problemas de aplicações web altamente escaláveis. Esse conjunto é conhecido como MEAN \textit{Stack}\footnote{\textit{Stack} no sentido de pilha, no caso uma pilha de aplicações.}, que é basicamente a utilização de quatro tecnologias baseadas em Javascript: \textbf{M}ongoDB, \textbf{E}xpress.js, \textbf{A}ngularJS e \textbf{N}ode.js. Será apresentado como o MEAN \textit{Stack} lida com a escalabilidade e como é o desempenho de alguns de seus componentes, através da análise de testes de desempenho.

Para demonstrar como uma aplicação MEAN é implementada e como seus componentes interagem entre si foi desenvolvida uma aplicação usando MEAN \textit{Stack} e o Socket.io  baseada na aplicação \textit{Math Race}. A idéia do \textit{Math Race} é realizar uma competição em tempo real para ver qual jogador acerta mais contas de matemáticas em determinado tempo. A partir desta implementação, testes de escalabilidade foram feitos para análise dos resultados.

\section{Organização do trabalho}

Este trabalho está dividido em quatro partes. A primeira parte trata os conceitos que foram utilizados para o desenvolvimento do tema, como aplicações web escaláveis, programação orientada a eventos, Javascript e NoSQL. 

Na segunda parte explicamos mais detalhadamente as tecnologias do MEAN que foram propostas, enfatizando algumas características que contribuem para a escalabilidade.

A terceira parte aborda como os componentes do MEAN são integrados, além de  mostrar um comparativo de uma aplicação MEAN com o LAMP. O LAMP é outra tecnologia bastante difundida, que utiliza o sistema Linux, o servidor Apache, o banco de dados MySQL e a linguagem de programação PHP\footnote{Podendo haver váriações com outras linguagens de programação como Python e Perl}. Os testes de desempenho também serão apresentados nesta parte.

A conclusão do trabalho e os possíveis trabalhos futuros são apresentados na quarta e última parte.