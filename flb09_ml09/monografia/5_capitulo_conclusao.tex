\chapter{Conclusão}
\label{conclusao}

  Esta monografia apresentou os conceitos e as tecnologias do MEAN \textit{Stack}, além de mostrar como é feita a integração destas ferramentas  para se criar uma aplicação web. Foram apresentados também os resultados de análises com aplicações implementadas neste ambiente, com o intuito de se alcançar uma conclusão sobre a escalabilidade através da utilização das ferramentas do MEAN \textit{Stack}.
  
 Através da análise dos resultados propostos foi concluído que o MEAN \textit{Stack} é de fato uma ótima opção para o desenvolvimento de aplicações web escaláveis, principalmente através de dois de seus componentes o Node.js e o MongoDB. O Angular.js e o Express.js atuam como facilitadores no processo de desenvolvimento, sem estes componentes, dependendo da complexidade da aplicação, o tempo gasto no desenvolvimento da aplicação do lado do cliente pode aumentar drasticamente, e a utilização do Node.js pode ser consideravelmente mais trabalhosa. 
  
%   Além das facilidades no desenvolvimento como por exemplo todos os componentes do MEAN reconhecerem o javascript.
   
\section{Trabalhos Futuros}
 Para os trabalhos futuros, seria interessante uma abordagem de maneira mais ampla, através da comparação do comportamento de diversas aplicações, com diferentes funcionalidades, e em diversos ambientes como Ruby e Python, além dos que foram utilizados nesta monografia. 