  In the last years, we have watched a big grow in the daily use of robotic systems. 
  More and more researchers in robotics have focus on the mobile robots development, they makes that 
  robots be capable of move and interact with the environment, what generates a huge field of new 
  applications and new challenges like: localization, map building and the problem of simultaneous
  localization and map building(SLAM).

  This work proposes a localization system for mobile robots based on the article \cite{wifiRadar}, 
  which provides a localization method of a mobile terminal in a indoor environment. The method is 
  based on the received signal strength(RSS), which is very used in localization techniques 
  in the wireless sensors networks due the easy applicability. Other techniques for localization 
  in the wireless sensors networks are shown in this work.
  
  The entire system was built on the Android plataform by the easy incorporation on robotics 
  due the big amount of sensors supported. 
  Some features of a Android application and of this operation system are brief commented. 
  
  The results got by the proposed system shown that:
      \begin{itemize}
      \item 22,8\% of the estimatives have precision of 1 meter or less.
      \item 77,1\% of the estimatives have precision of 3 meter or less.
      \item 91,4 \% of the estimatives have precision of 5 meter or less.
     \end{itemize}
  \textbf{Keywords:} RSS, Localization, RSS fingerprint, Mobile Robots, Android, WSNs.